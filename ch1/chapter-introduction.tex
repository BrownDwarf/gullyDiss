\chapter{Introduction}


\section{Overview}

\section{Silicon grisms for the James Webb Space Telescope NIRCam}

The James Webb Space Telescope (JWST) will be the flagship NASA astrophysics mission of this decade.  Its primary science goals include the detection of the first galaxies after reionization, and characterization of exoplanetary atmospheres.  These and other science goals required the development of new technologies, including microshutter arrays, lightweight cryogenic mirrors, infrared detectors, and high-precision wavefront sensing.  In Chapter \ref{chSPIE_2010}, I describe one of these technologies we have developed and tested at UT Austin, the silicon \emph{grism}.

These devices have exquisite optical characteristics: phase surfaces flat to $\lambda$/100 peak to valley at the blaze wavelength, diffraction-limited PSFs down to $10^{-5}$ of the peak, low scattered light levels, and large resolving-power slit-width products for their width and thickness.  The one possible drawback to these devices is the large Fresnel loss caused by the large refractive index of Si.  Chapter \ref{chSPIE_2010} reports on throughput and phase-surface measurements for a sample grating with a high performance antireflection coating on both the flat and grooved surfaces. These results indicate that we can achieve very high on-blaze efficiencies.  The high throughput should make Si grisms an attractive dispersive element for moderate resolution IR spectroscopy in both ground and space based instruments throughout the 1.2-8 $\mu$m spectral region.

NIRCam will help to answer some key questions about star and planet formation.  For example, ``How do cloud cores collapse to form isolated protostars?''.  The Si grisms discussed in Chapter \ref{chSPIE_2010} can facilitate answering this question by enabling grain composition studies with grism spectroscopy \citep{2012SPIE.8442E..2NB}.

The NIRCam Si grisms offers a unique capability for transit spectroscopy.  The wavelength range 2.4$-$5.0 $\mu$m has numerous spectral feaures of interest for discriminating between planetary atmosphere models.  For example, the 2.4$-$4.0 $\mu$m transit spectrum of GJ436b will be able to distinguish between models with and without non-equilibrium photochemical hydrocarbon products from insolation from GJ436 \citep{2011ApJ...727...65S}.  The JWST grism spectrum of GJ436b will have detectable signatures of CH$_4$, HCN, C$_2$H$_2$, CO, and CO$_2$.

NIRCam has recently (November 2014) completed its second cryogenic vacuum test, and is performing as expected.

Si grisms like those in Chapter \ref{chSPIE_2010} served as pathfinders for Si immersion echelle grating.  The precision fabrication of Si immersion gratings is more demanding than it is for Si grisms.  The heightened precision demands for immersion gratings over grisms rises from two effects.  First, grisms are operated in transmission, whereas echelles operate in reflection.  Echelles therefore require $2n/(n-1) = 2.8$ times less wavefront error to meet the same optical performance as Si grisms.  Additionally, Si echelles operate at much steeper blaze angles than Si grisms do.  For example, an R3 echelle will will be 8.8 times more sensitive to wavefront error than its grism counterpart with a blaze angle of 6.16$^\circ$.  Taking the reflection and blaze angle effects together, we see that Si immersion echelle gratings require 25 times more precision than Si grisms of the same aperture and design wavelength.  

Despite their fabrication challenges, Silicon immersion gratings are attractive because they offer size and cost savings for high-resolution near-infrared spectrographs.  The Immersion GRating Infrared Spectroph (IGRINS) \cite{} achieves spectral resolution $R=\frac{\lambda}{\Delta \lambda} = 40,000$ simultaneously over H and K near-infrared band atmospheric windows (1.5$-$2.5 $\mu$m).  

In Chapter \ref{chapter_CA1}, I chronicle the metrology of the R3 silicon immersion echelle grating for IGRINS.  The grating is 30 $\times$ 80 mm, etched into a monolithic silicon prism.  Optical interferometry of the grating surface in reflection indicates high phase coherence ($< \lambda/6$ peak to valley surface error over a 25 mm beam at $\lambda=632$ nm).  Optical interferometry shows small periodic position errors of the grating grooves.  These periodic errors manifest as spectroscopic ghosts.  High dynamic range monochromatic spectral purity measurements reveal ghost levels relative to the main diffraction peak at $1.6 \times 10^{-3}$ at $\lambda = 632$ nm in reflection, consistent with the interferometric results.  Improved grating surfaces demonstrate reflection-measured ghosts at negligible levels of $10^{-4}$ of the main diffraction peak.  Relative on-blaze efficiency is $\sim$75\%.  We investigate the immersion grating blaze efficiency performance over the entire operational bandwidth $1500 < \lambda(nm) < 2500$ at room temperature.

IGRINS has been in opertation for over 1 year, with \emph{no performance limitation attributable to the immersion grating}.  Recently, \citet{2015ATel.6901....1P} have shown that 2MASS J06593158$-$0405277 shows a similar spectrum to outbursting \emph{FU Orionis} objects.  The wide spectral bandwidth and high resolution stand to revolutionize studies of protostars.  The construction of a panchromatic spectral atlas of dwarfs, giants, and young stars is underway (IGRINS collaboration, \emph{in prep.}).  An especially intriguing possibility is to perform robust spectroscopic inference on hundreds or thousands of IGRINS spectra to construct semi-empircal stellar spectra with imperfect models \citep{2014arXiv1412.5177C}.  The unique science capabilities of IGRINS rely upon the high performance of the silicon immersion grating.

Other astronomical instruments are using Si immersion gratings.  The NASA Infrared Telescope Facility (IRTF) will soon commission iSHELL \citep{2012SPIE.8446E..2CR}, which will reach spectral resolution $R=70,000$, with spectral range 1.15$-$5.4 $\mu$m.  The performance demands for the iSHELL immersion grating are even higher than those for the IGRINS immersion grating, since the IRTF site has lower median seeing conditions than McDonald, and iSHELL will reach shorter wavelength than IGRINS.  These two effects conspire to make the fabrication of the iSHELL immersion grating more demanding than the IGRINS immersion grating.  In Chapter \ref{chap_ebeam} I describe my work to improve the precision of Si immersion grating production.

The grooves in Si gratings are made with semiconductor lithography techniques, to date almost entirely using contact mask photolithography.  Planned near-infrared astronomical spectrographs like iSHELL require either finer groove pitches or higher positional accuracy than standard UV contact mask photolithography can reach.  A collaboration between the University of Texas at Austin Silicon Diffractive Optics Group and the Jet Propulsion Laboratry Microdevices Laboratory has experimented with direct writing silicon immersion grating grooves with electron beam lithography.  The patterning process involves depositing positive e-beam resist on 1 to 30 mm thick, 100 mm diameter monolithic crystalline silicon substrates.  We then use the facility JEOL 9300FS e-beam writer at JPL to produce the linear pattern that defines the gratings.

There are three key challenges to produce high-performance e-beam written silicon immersion gratings.  (1) E-beam field and subfield stitching boundaries cause periodic cross-hatch structures along the grating grooves.   The structures manifest themselves as spectral and spatial dimension ghosts in the diffraction limited point spread function (PSF) of the diffraction grating.  In Chapter \ref{chap_ebeam}, we show that the effects of e-beam field boundaries must be mitigated.  We have significantly reduced ghost power with only minor increases in write time by using four or more field sizes of less than 500 $\mu$m. (2) The finite e-beam stage drift and run-out error cause large-scale structure in the wavefront error.  We deal with this problem by applying a mark detection loop to check for and correct out minuscule stage drifts.  We measure the level and direction of stage drift and show that mark detection reduces peak-to-valley wavefront error by a factor of 5. (3) The serial write process for typical gratings yields write times of about 24 hours- this makes prototyping costly.  We discuss work with negative e-beam resist to reduce the fill factor of exposure, and therefore limit the exposure time.
We also discuss the tradeoffs of long write-time serial write processes like e-beam with UV photomask lithography.  We show the results of experiments on small pattern size prototypes on silicon wafers.  Current prototypes now exceed 30 dB of suppression on spectral and spatial dimension ghosts compared to monochromatic spectral purity measurements of the backside of Si echelle gratings in reflection at 632 nm.  We perform interferometry at 632 nm in reflection with a 25 mm circular beam on a grating with a blaze angle of 71.6$^\circ$.  The measured wavefront error is 0.09 waves peak to valley.

The next generation of high precision near-IR spectographs employing silicon immersion gratings will push further the demands on optical performance.  The Giant Magellan Telescope Near Infrared Spectrograph (GMTNIRS) will have 5 separate spectrograph channels, each with its own custom Si immersion grating and IR focal plane array \citep{2014SPIE.9147E..22J}.  The scientific requirements drive high spectral resolution ($R\sim100,000$) for the long wavelength channels (3$-$5.4 $\mu$m).  The design calls for a Si immersion grating between 150 and 200 mm in illuminated length- larger than our existing 100 mm immersion grating substrates.  The substrates will also have to exceed 30 mm in thickness, which is larger than what our existing fabrication equipment can handle.  In Chapter \ref{ch_SiGaps}, I demonstrate a technique to directly bond patterned Si wafers to optical Si prisms, alleviating the need for patterning and processing on monolithic Si substrates.


\section{Bonded Si wafers}

\begin{itemize}
	\item Bonding wafers will make it easier to make gratings
	\item Summary of the chapter
	\item Applicability to earth-science satellites and astronomy
\end{itemize}


\section{Allers et al. sspectroscopic sample}

\begin{itemize}
	\item Explanation of Allers et al. 2006 sample
	\item Low-mass star formation and the IMF
	\item Brief summary of the short chapter
	\item Lay the ground work for the next chapter
\end{itemize}

\section{Discovery and characterization of diskless young brown dwarfs}

\begin{itemize}
	\item Summary of the chapter
\end{itemize}


\section{Extras/Appendices}

\begin{itemize}
	\item Crowbar
\end{itemize}

\section{Conclusions}

Technology and ideas catalyze discovery.  Astronomical technology development started 405 years ago with Galileo's 2 inch telescope and \emph{Sidereus Nucius}.  Today there are sixteen optical telescopes with diameters of 6.5 meters or larger, and there are IPython Jupyter Notebooks\footnote{\url{https://jupyter.org/}}\footnote{Galileo's detailed notebooks of Jupiter contained observational data, calculations, figures, and explanatory material, and were, in part, the inspiration for the name of Jupyter Notebooks.}  This PhD dissertation  pushes the frontier of both scientific technology and observational astronomy.  There are two main threads.  First, I developed silicon diffractive optics for high resolution near infarared astronomical spectroscopy.  The devices leverage high precision silicon semiconductor lithography to provide optical quality diffraction grooves.  
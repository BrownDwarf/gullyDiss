\chapter{Introduction}

\section{Innovative Technologies for Star and Planet Formation}

The James Webb Space Telescope (JWST) will be the flagship NASA astrophysics mission of this decade.  Its primary science goals include the detection of the first galaxies after reionization, and characterization of exoplanetary atmospheres.  These and other science goals required the development of new technologies, including microshutter arrays, lightweight cryogenic mirrors, infrared detectors, and high-precision wavefront sensing.  In Chapter \ref{chSPIE_2010}, I describe one of these technologies we have developed and tested at UT Austin, the silicon \emph{grism}, destined for the NIRCam instrument \citep{Greene10} on the JWST.

These devices have exquisite optical characteristics: phase surfaces flat to $\lambda$/100 peak to valley at the blaze wavelength, diffraction-limited PSFs down to $10^{-5}$ of the peak, low scattered light levels, and large resolving-power slit-width products for their width and thickness.  The one possible drawback to these devices is the large Fresnel loss caused by the large refractive index of Si.  Chapter \ref{chSPIE_2010} reports on throughput and phase-surface measurements for a sample grating with a high performance antireflection coating on both the flat and grooved surfaces. These results indicate that we can achieve very high on-blaze efficiencies.  The high throughput should make Si grisms an attractive dispersive element for moderate resolution IR spectroscopy in both ground and space based instruments throughout the 1.2-8 $\mu$m spectral region.


The NIRCam Si grisms offer a unique capability for spectroscopy of the atmospheres of extra-solar planets.  The composition of the extrasolar planetary atmosphere can be ascertained from spectroscopy of the planet in- and out- of transit \citep{2008ApJ...673L..87R}, when the planet passes in front of the star as viewed from Earth.  The Si grisms can be used for \emph{slitless} transit spectroscopy, which eliminates the ambiguity of slit-losses when disentangling the in- and out- of transit signals.  The wavelength range 2.4$-$5.0 $\mu$m has numerous spectral feaures of interest for discriminating between planetary atmosphere models.  For example, the 2.4$-$4.0 $\mu$m transit spectrum of GJ436b will be able to distinguish between models with and without non-equilibrium photochemical hydrocarbon products from insolation from GJ436 \citep{2011ApJ...727...65S}.  The JWST grism spectrum of GJ436b could have detectable signatures of CH$_4$, HCN, C$_2$H$_2$, CO, and CO$_2$.  NIRCam has recently (November 2014) completed its second cryogenic vacuum test, and is performing as expected.

Si grisms like those in Chapter \ref{chSPIE_2010} served as pathfinders for Si immersion echelle gratings.  The precision fabrication of Si immersion gratings is more demanding than it is for Si grisms.  The heightened precision demands for immersion gratings over grisms rises from two effects.  First, grisms are operated in transmission, whereas echelles operate in reflection.  Echelles therefore require $2n/(n-1) = 2.8$ times less wavefront error to meet the same optical performance as Si grisms.  Additionally, Si echelles operate at much steeper blaze angles than Si grisms do.  For example, an R3 echelle will will be 8.8 times more sensitive to wavefront error than its grism counterpart with a blaze angle of 6.16$^\circ$.  Taking the reflection and blaze angle effects together, we see that Si immersion echelle gratings require 25 times more precision than Si grisms of the same aperture and design wavelength.  

Despite their fabrication challenges, Silicon immersion gratings are attractive because they offer size and cost savings for high-resolution near-infrared spectrographs.  The Immersion GRating Infrared Spectrograph (IGRINS) \cite{2014SPIE.CHANPARK.IGRINS} achieves spectral resolution $R=\frac{\lambda}{\Delta \lambda} = 40,000$ simultaneously over H and K near-infrared band atmospheric windows (1.5$-$2.5 $\mu$m).  

In Chapter \ref{chapter_CA1}, I chronicle the metrology of the R3 silicon immersion echelle grating for IGRINS.  The grating is 30 $\times$ 80 mm, etched into a monolithic silicon prism.  Optical interferometry of the grating surface in reflection indicates high phase coherence ($< \lambda/6$ peak to valley surface error over a 25 mm beam at $\lambda=632$ nm).  Optical interferometry shows small periodic position errors of the grating grooves.  These periodic errors manifest as spectroscopic ghosts.  High dynamic range monochromatic spectral purity measurements reveal ghost levels relative to the main diffraction peak at $1.6 \times 10^{-3}$ at $\lambda = 632$ nm in reflection, consistent with the interferometric results.  Improved grating surfaces demonstrate reflection-measured ghosts at negligible levels of $10^{-4}$ of the main diffraction peak.  Relative on-blaze efficiency is $\sim$75\%.  We investigate the immersion grating blaze efficiency performance over the entire operational bandwidth $1500 < \lambda(nm) < 2500$ at room temperature.

IGRINS has been in operation for over 1 year on the 2.7 m Harlan J. Smith Telescope at McDonald Observatory, with \emph{no performance limitation attributable to the immersion grating}.  Recently, \citet{2015ATel.6901....1P} have shown that 2MASS J06593158$-$0405277 shows a similar spectrum to outbursting \emph{FU Orionis} objects.  The wide spectral bandwidth and high resolution stand to revolutionize studies of protostars.  The construction of a panchromatic spectral atlas of dwarfs, giants, and young stars is underway (IGRINS collaboration, \emph{in prep.}).  An especially intriguing possibility is to perform robust spectroscopic inference on hundreds or thousands of IGRINS spectra to construct semi-empirical stellar spectra with imperfect models \citep{2014arXiv1412.5177C}.  The unique science capabilities of IGRINS rely upon the high performance of the silicon immersion grating.

Other astronomical instruments are using Si immersion gratings.  The NASA Infrared Telescope Facility (IRTF) will soon commission iSHELL \citep{2012SPIE.8446E..2CR}, which will reach spectral resolution $R=70,000$, with spectral range 1.15$-$5.4 $\mu$m.  The performance demands for the iSHELL immersion grating are even higher than those for the IGRINS immersion grating, since the IRTF site, Mauna Kea, has lower median seeing conditions than McDonald, and iSHELL will reach shorter wavelengths than IGRINS.  These two effects conspire to make the fabrication of the iSHELL immersion grating more demanding than the IGRINS immersion grating.  In Chapter \ref{chap_ebeam}, I describe my work to improve the precision of Si immersion grating production.

The grooves in Si gratings are made with semiconductor lithography techniques, to date almost entirely using contact mask photolithography.  Planned near-infrared astronomical spectrographs like iSHELL require either finer groove pitches or higher positional accuracy than standard UV contact mask photolithography can reach.  A collaboration between the University of Texas at Austin Silicon Diffractive Optics Group and the Jet Propulsion Laboratry Microdevices Laboratory has experimented with direct writing silicon immersion grating grooves with electron beam lithography.  The patterning process involves depositing positive e-beam resist on 1 to 30 mm thick, 100 mm diameter monolithic crystalline silicon substrates.  We then use the facility JEOL 9300FS e-beam writer at JPL to produce the linear pattern that defines the gratings.

There are three key challenges to produce high-performance e-beam written silicon immersion gratings.  (1) E-beam field and subfield stitching boundaries cause periodic cross-hatch structures along the grating grooves.   The structures manifest themselves as spectral and spatial dimension ghosts in the diffraction limited point spread function (PSF) of the diffraction grating.  In Chapter \ref{chap_ebeam}, we show that the effects of e-beam field boundaries must be mitigated.  We have significantly reduced ghost power with only minor increases in write time by using four or more field sizes of less than 500 $\mu$m. (2) The finite e-beam stage drift and run-out error cause large-scale structure in the wavefront error.  We deal with this problem by applying a mark detection loop to check for and correct out minuscule stage drifts.  We measure the level and direction of stage drift and show that mark detection reduces peak-to-valley wavefront error by a factor of 5. (3) The serial write process for typical gratings yields write times of about 24 hours- this makes prototyping costly.  We discuss work with negative e-beam resist to reduce the fill factor of exposure, and therefore limit the exposure time.
We also discuss the tradeoffs of long write-time serial write processes like e-beam with UV photomask lithography.  We show the results of experiments on small pattern size prototypes on silicon wafers.  Current prototypes now exceed 30 dB of suppression on spectral and spatial dimension ghosts compared to monochromatic spectral purity measurements of the backside of Si echelle gratings in reflection at 632 nm.  We perform interferometry at 632 nm in reflection with a 25 mm circular beam on a grating with a blaze angle of 71.6$^\circ$.  The measured wavefront error is 0.09 waves peak to valley.

The next generation of high precision near-IR spectrographs employing silicon immersion gratings will push further the demands on optical performance.  

The Giant Magellan Telescope Near Infrared Spectrograph (GMTNIRS) will have 5 separate spectrograph channels, each with its own custom Si immersion grating and IR focal plane array \citep{2014SPIE.9147E..22J}.  The scientific requirements drive high spectral resolution ($R\sim100,000$) for the long wavelength channels (3$-$5.4 $\mu$m).  The design calls for a Si immersion grating between 150 and 200 mm in illuminated length- larger than our existing 100 mm immersion grating substrates.  The substrates will also have to exceed 30 mm in thickness, which is larger than what our existing fabrication equipment can handle.  

In Chapter \ref{ch_SiGaps}, I demonstrate a technique to directly bond Si wafers to optical Si prisms, alleviating the need for patterning and processing on monolithic Si substrates.

Silicon direct bonding offers flexibility in the design and development of Si optics by allowing manufacturers to combine subcomponents with a potentially lossless and mechanically stable interface. The bonding process presents challenges in meeting the requirements for optical performance because air gaps at the Si interface cause large Fresnel reflections. Even small (35 nm) gaps reduce transmission through a direct bonded Si compound optic by 4\% at $\lambda = 1.25 \; \mu$m at normal incidence. Understanding the optical effects of such gaps and evaluation of various methods for preventing or eliminating them demands that we have a method for measuring not only the two-dimensional extent of the gaps but also their axial extent.  Existing methods for studying gaps provide only poor measures of this axial extent.  We describe a bond inspection method that makes use of precision slit spectroscopy.  Using spectroscopy, we detect and measure gaps as small as 14 nm by measuring transmission as a function of wavelength. Our analysis involves modeling multiple incoherent reflections within the substrate interfaces with a wave transfer matrix, modified for intensities rather than complex amplitudes. We infer the gap axial extent and fill factor with Monte Carlo Markov Chain (MCMC) and a flexible Gaussian process noise model.  We describe the measurement and analysis techniques and demonstrate the validity of the approach by measuring bond gaps of known depths produced by microlithography.

%Spectroscopy of planets with GMTNIRS

The innovations in silicon immersion grating technology that I have pioneered center around astronomical instruments like JWST NIRCam, IGRINS, iSHELL, and GMTNIRS.  These technologies could also benefit the Earth Science community, in which remote sensing of trace gases in the Earth's atmosphere is a major priority.  In particular, a future space-based mission like the Orbiting Carbon Observatory will associate local sources and sinks of greenhouse gases.  The scientific requirements of high spatial and spectral resolution drive a ``pushbroom'' spectrograph design, with near-IR capability to detect methane.  This spectrograph design differs from those in astronomical applications because of its much larger angular slit length, and its higher groove frequency.  The innovations in direct Si$-$Si bonding and metrology described in Chapter \ref{ch_SiGaps} will facilitate the transfer of technologies between the fields of astronomy and earth science.  

\section{Observational Studies of Star and Planet Formation}

One outstanding question in star formation is ``How do the properties of young stars affect the evolution of their circumstellar disks''.  The effects of stellar mass, multiplicity, and radiative properties have been studied across the mass range from Herbig Ae-Be stars ($2-8 M_{\odot}$) to brown dwarfs ($<0.08 M_{\odot}$) \citep{2011ARA&A..49...67W}.  Extending studies further into the substellar regime is especially appealing since brown dwarfs offer vastly different radiative environments and lower accretion rates \citep{2003ApJ...592..266M,2009ApJ...696.1589H}.

The current prevailing opinion is that most brown dwarfs form just like stars \citep{2014prpl.conf..619C}.  The existence of wide separation, similar mass binary brown dwarfs \citep{2004ApJ...614..398L,2006PhDT.........2A,2007ApJ...660.1492C,2009ApJ...691.1265L} shows that at least some brown dwarfs form from direct gravitational collapse of a gas cloud, and could not have been formed from gravitational instability in the disk of a more massive star.  Circumstellar disks are common around young ($\sim 1$ Myr) brown dwarfs \citep{2005ApJ...635L..93L,allers06,2012ARA&A..50...65L,2013A&A...559A.126A}.  Large disks cannot survive a three-body interaction that would cause ejection from a stellar counterpart, while small protolunar disks could survive ejection \citep{2009MNRAS.392..413S}.  These wimpy disks would be outwardly truncated at a semimajor axis of tens of AU \citep{2009MNRAS.392..413S}.  These outwardly truncated disks would demonstrate an observational signature of dramatically suppressed mid-IR and far-IR excess emission.  The abundance of disk-bearing young brown dwarfs is inconsistent with a large population of outwardly truncated disks, pointing towards formation like stars as the dominant formation pathway for brown dwarfs \citep{2012ARA&A..50...65L}.  Additionally, recent interferometric ALMA continuum visibilities demonstrate that 3 Taurus brown dwarfs have large, $\sim70 AU$ disks \citep{2014ApJ...791...20R}.  See \citet{2014prpl.conf..619C} for more lines of evidence, and a discussion of the IAU definition of brown dwarfs and giant planets.

Because they share a common formation pathway with stars, brown dwarfs must also share similar pre-main sequence evolution characteristics.  Young stars go through a set of clear evolutionary stages that have distinctive appearances and physics.  Brown dwarfs must form in exceptionally dense molecular cores, then proceed through an envelope phase, a disk-bearing phase, and a dissipation phase.  The observational signatures of these stages correspond to the observational classes 0, I, II, and III \citep{1987ApJ...312..788A}.  Stars have been observed in all of these stages \citep{2014prpl.conf..195D} while brown dwarfs have been observed in classes I, II, and III.  A few candidate class 0 proto-brown dwarfs exist\citep{2012Sci...337...69A,2014A&A...564A..32P,2014MNRAS.444..833P}, for example the source IC348-SMM2E is consistent with an envelope mass of merely $\sim35 M_{Jup}$.  The pre-main sequence classes are based on their $T_{\mathrm{bol}}$ \citep{1993ApJ...413L..47M}, the spectral index in the mid-infrared \citep{1984ApJ...287..610L} and, for class I-III on H$\alpha$ equivalent widths (accretion rates).  The dust in the disk is much easier to detect at mid-IR wavelengths, since dust has most of the opacity in the disk, despite having $\lesssim1\%$ of the mass of the disk.  

Strong H emission lines indicate mass accretion, and should be present in all observational classes, except possibly class III (\emph{e.g.} weak-lined \emph{T-Tauri} star analogs) \citep{2009apsf.book.....H}.  Accretion is accompanied by outflows- spatially resolved collisionally excited lines like [OI] $\lambda 6300, 6363 \angstrom$ and [SII] $\lambda 6731 \angstrom$ \citep{2005Natur.435..652W} indicate shocked gas.  Photometric variability, a hallmark of YSOs at all classes and all wavelengths \citep{2014AJ....148...92R,2014AJ....147...82C}, is evidence for ongoing accretion, photospheric hot spots, and dynamic disk processes.

We assume that the majority of brown dwarfs form like stars, but that an unknown minority may have formed by ejection from a protoplanetary disk.

While you might expect accretion and outflows to be present among brown dwarfs, there are some differences from what we see in stellar mass objects of comparable age.  The radiative environment will be much softer around brown dwarfs, since the accretion luminosity $L_{\mathrm{acc}}$ will scale down with the low mass but large radii of young brown dwarfs, with observations indicating brown dwarfs have about $10^{-3}-10^{-4}$ times the accretion luminosity of their $T-Tauri$ counterparts that have only 10 times the mass \citep{2004A&A...424..603N}.  Young brown dwarf emission peaks at around 1 $\mu$m, almost twice that of their solar mass counter parts.  Passively irradiated disk models \citep{2003A&A...398..607D} show that the sequence of Herbig Ae/Be to \emph{T-Tauri} to young brown dwarf forms a continuum, with no significant variations in the turbulent mixing strength \citep{2012A&A...539A...9M}. However many factors affect the appearance of the SED.  Chief among the factors are grain growth and dust settling \citep{2010ApJ...712..925C}.  The key idea is that the central object temperature and luminosity controls the size of the dust-free cavity through photoevaporation \citep{2011ARA&A..49...67W}.  The location in the disk where the equilibrium temperature of a dust grain is 1400 K is roughly equal to the dust inner rim radius, $R_{inner}$ \citep{2001ApJ...560..957D}.  The detailed location of the dust inner rim depends on the grain size, composition, and local pressure \citep{2005A&A...438..899I}, and presence or absense of disruptive planets \citep{2014prpl.conf..667B,2015arXiv150302649P}.  One main open question is ``What happens to the dust inner rim when the brown dwarf photospheric temperature declines to equal the dust sublimation temperature?''  Can dust survive all the way up to the stellar surface?  The luminosity of the dust inner rim depends on both its equilibrium temperature and emitting area.  Assuming a fixed (1400 K) dust inner rim temperature, the emitting area of the dust rim controls the luminosity of the disk compared to the luminosity of the star.  The ratio of disk flux to photospheric flux at a given wavelength will therefore decrease as a function of photospheric temperature.  This effect has been used to explain the high frequency of objects with spectral energy distributions similar to those of transition disks without invoking a region in the inner disk where the temperatures reach the sublimation temperature but where dust is depleted \citep{2009MNRAS.394L.141E}, and is consistent with \emph{Herschel} observations towards brown dwarfs in \emph{Ophiuchus} \citep{2013A&A...559A.126A}.


The average steady state accretion rates scale as a function of mass with the scaling factor $\alpha$ in $\dot{M}\propto M^\alpha$ approximately equal to $\alpha \sim 2$ \citep{2008A&A...481..423G,2006ApJ...639L..83A,2006A&A...452..245N,2009ApJ...696.1589H}.  The explanation for $\alpha \sim 2$ is unclear \citep{2006ApJ...648..484H}.  The ratio of stellar mass to stellar radius $M/R$ scales with mass \citep{2009AIPC.1094..102C}, so the energy of impact $\propto M/R$ should also scale with mass.  Existing measures of accretion rate are noisy owing to the uncertainty in assigning mass accretion rates based on emission line strengths, and the intrinsic variability of accretion.  Instruments with pan-chromatic UV-visible-near-IR spectroscopy \citep{2011A&A...536A.105V,2012A&A...548A..56R} are improving the accuracy of these methods.  It has recently been suggested that the scaling factor could arise from selection effects of brown dwarfs \citep{2010MNRAS.409.1307M}.  If true, this selection bias would mean there is an unobserved population of luminous brown dwarfs with mass accretion rates $\log{\dot{M}/(M_{\odot}\mathrm{yr}^{-1})} >-9$ that has largely evaded detection.

Brown dwarfs will not follow a Henyey track \citep{1959ApJ...129....2H}.  Instead, brown dwarfs monotonically cool and dim for billions of years.  For example, a 50 $M_{Jup}$ brown dwarf will decrease in luminosity by a factor of about 2 over the $\lesssim5$ Myr disk lifetime \citep{2002A&A...382..563B}.

For low mass stars, the majority of mass accretion occurs in stages 0 and I, when the protostar is most embedded \citep{2007ARA&A..45..565M,2014prpl.conf..195D}.  The ubiquity of underluminous class 0 YSOs indicates that most mass accretes in short-lived episodes \citep{2008ApJS..179..249D}.  Mass accretion rates are highly variable during this stage \citep{2006ApJ...650..956V,2009ApJ...702L..27B,2014prpl.conf..387A}.  Specifically, the \emph{Spitzer} legacy program ``From molecular cores to planet-forming disks'' \citep[][\emph{c2d}]{2009ApJS..181..321E} pointed out that a constant mass accretion rate of $10^{-6} M_{\odot}\mathrm{yr}^{-1}$ would form a solar mass star in 1 Myr, but that accretion luminosities corresponding to $\sim10^{-8} M_{\odot}\mathrm{yr}^{-1}$ were observed.  So protostars must accrete their mass in several very short bursts, with burst timescales consistent with \emph{FU Orionis} events \citep{2014prpl.conf..195D}.  The presence of chemical species indicative of high disk temperatures also lends support to the episodic accretion paradigm \citep{2012ApJ...754L..18V}.  Accretion history is predicted to have an observational consequence on the luminosities of very low mass stars and brown dwarfs \citep{2009ApJ...702L..27B,2011ApJ...730...32S}.  Specifically, episodic accretion onto a 50 $M_{Jup}$ brown dwarf is predicted to demonstrate a luminosity spread equivalent to an age spread of $\sim10$ Myr.  Sources that look young are young, while sources that look old, could still be young.  The effect is greatest for the lowest mass sources, which can be understood because these sources are accreting a large fraction of their mass in these episodes.  Young (1-10 Myr) brown dwarfs offer a unique laboratory to directly probe the evolutionary history of individual objects, since the appearance of a luminosity spread mimicking an age spread could indicate that episodic accretion has occurred.  This indicator of episodic accretion would be complementary to the evidence for episodic accretion coming from the underluminosity problem in Class 0 protostars.


In Chapter \ref{ch_CS16}, we confirm and characterize, using $R\sim 2000$ near-infrared spectroscopy, 17 candidate young substellar objects in the Ophiuchus, Lupus I, and Chamaeleon II star forming regions from the Allers et al. 2006 sample.  The spectral types range from M2$-$L2.5, with extinctions of $0 < A_V < 10$.  We assign youth based on the presence of mid-IR excess and gravity sensitive spectral indices, like the triangular $H-$band continuum shape and the $\mathrm{Na\, I}$ equivalent width.  All but one source from the photometrically selected sample of 19 are confirmed as bona fide young late type objects, an exceptional confirmation rate for this type of study.  Two of the sources are known wide-separation binaries.  We explore the advantage of increased resolution over existing low resolution spectroscopy for six objects.  The objects' derived luminosities are in the range $-3.3 < \log L/L_{\odot} < -0.6$, placing the lowest luminosity candidates comfortably in the substellar or even sub brown-dwarf mass regime.

\citet{allers06} selected their sources for demonstrating the presence of mid-IR excess indicative of a circumstellar disk.  In Chapter 7, I search the same region of \citet{allers06} towards \emph{Ophiuchus} for the presence of young brown dwarfs which do not exhibit mid-IR excess indicative of a circumstellar disk.  A custom $W-$band filter centered on 1.4 $\mu$m is pivotal in distinguishing background contaminants from young brown dwarfs with intrinsic water absorption in their photospheres.  Since this selection strategy is unbiased towards presence of mid-IR excess, we can pick out the diskless young brown dwarfs.  These diskless young brown dwarfs are Class III analogs that have already undergone a phase of disk dispersal.  The dominant mechanism of disk dispersal is still unknown, though planet formation, binary companions, and photoevaporation may all play a role.  Whatever the mechanism, it is clear that the timescale for disk clearing is short compared to the disk lifetime.  Out of 15 sources towards this region we find that only one source exhibits mid-infrared colors indicative of a transition disk.  The small number of sources and possible contamination towards Upper Scorpius make it challenging to draw strong conclusions on the disk fraction towards this region.  Our estimated disk fraction of 33\% is lower than but consistent with estimates of other young star forming regions.

\section{Summary}
This work has developed the technology with which the next generation of astronomical discoveries will be made.  At the same time I have used new selection strategies to discover a population of young low mass stars and brown dwarfs with evolved disks or no disks.

\chapter{Introduction}


\section{Overview}

\section{Silicon grisms for the James Webb Space Telescope NIRCam}

The James Webb Space Telescope (JWST) will be the flagship NASA astrophysics mission.  Its primary science goals include the detection of the first galaxies after reionization, and characterization of exoplanetary atmospheres.  These and other science goals required the development of new technologies, including microshutter arrays, lightweight cryogenic mirrors, infrared detectors, and high-precision wavefront sensing.  In Chapter \ref{chSPIE_2010}, I describe one of these technologies we have developed and tested at UT Austin, the silicon \emph{grism}.

\begin{itemize}
	\item What is NIRcam going to do
	\item Exoplanet transit spectroscopy
	\item Summary of the chapter
	\item Current status of NIRCam
\end{itemize}  

These devices have exquisite optical characteristics: phase surfaces flat to $\lambda$/100 peak to valley at the blaze wavelength, diffraction-limited PSFs down to $10^{-5}$ of the peak, low scattered light levels, and large resolving-power slit-width products for their width and thickness.  The one possible drawback to these devices is the large Fresnel loss caused by the large refractive index of Si.  Chapter \ref{chSPIE_2010} reports on throughput and phase-surface measurements for a sample grating with a high performance antireflection coating on both the flat and grooved surfaces. These results indicate that we can achieve very high on-blaze efficiencies.  The high throughput should make Si grisms an attractive dispersive element for moderate resolution IR spectroscopy in both ground and space based instruments throughout the 1.2-8 $\mu$m spectral region.


\section{The IGRINS immersion grating}

\begin{itemize}
	\item Connect grisms to immersion gratings
	\item Summary of the chapter
	\item Current status of IGRINS
	\item Examples of Science IGRINS has been doing
\end{itemize}

\section{Electron-beam lithography of immersion gratings}

\begin{itemize}
	\item Immersion gratings for iSHELL and GMTNIRS
	\item E-beam lithography compared to photolithography
	\item Current status
\end{itemize}

\section{Bonded Si wafers}

\begin{itemize}
	\item Bonding wafers will make it easier to make gratings
	\item Summary of the chapter
	\item Applicability to earth-science satellites and astronomy
\end{itemize}

\section{Allers et al. sspectroscopic sample}

\begin{itemize}
	\item Explanation of Allers et al. 2006 sample
	\item Low-mass star formation and the IMF
	\item Brief summary of the short chapter
	\item Lay the ground work for the next chapter
\end{itemize}

\section{Discovery and characterization of diskless young brown dwarfs}

\begin{itemize}
	\item Summary of the chapter
\end{itemize}


\section{Extras/Appendices}

\begin{itemize}
	\item Crowbar
\end{itemize}

\section{Conclusions}

Technology and ideas catalyze discovery.  Astronomical technology development started 405 years ago with Galileo's 2 inch telescope and \emph{Sidereus Nucius}.  Today there are sixteen optical telescopes with diameters of 6.5 meters or larger, and there are IPython Jupyter Notebooks\footnote{\url{https://jupyter.org/}}\footnote{Galileo's detailed notebooks of Jupiter contained observational data, calculations, figures, and explanatory material, and were, in part, the inspiration for the name of Jupyter Notebooks.}  This PhD dissertation  pushes the frontier of both scientific technology and observational astronomy.  There are two main threads.  First, I developed silicon diffractive optics for high resolution near infarared astronomical spectroscopy.  The devices leverage high precision silicon semiconductor lithography to provide optical quality diffraction grooves.  
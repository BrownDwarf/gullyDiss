\chapter{Evolved disk population of young brown dwarfs towards \emph{Ophiuchus}}

%------------------------------------------------------------------------------------------
\section{Introduction}
%------------------------------------------------------------------------------------------

How do the circumstellar disks around young brown dwarfs differ from those around their higher mass \emph{T-Tauri} star counterparts?  One emerging answer to this question is that the disk frequency and lifetime may be a function of stellar mass and star forming region.  At an age of $\sim$1 Myr, 40\% of the brown dwarfs in the Taurus region have disks, which is lower than the disk fraction for Taurus stars \citep{2006ApJ...645..676L}.  While that might suggest that fewer brown dwarfs are born with disks, samples in the older, $\sim$2$-$3 Myr, Chamaeleon and IC 348 regions complicate the story.  \citet{2005ApJ...631L..69L} find that $\sim$50\% of those brown dwarfs have disks, slightly higher than the fraction for stars.  \citet{2008ApJ...681.1584R} showed that 3/5 brown dwarfs in the TW Hydrae Association retained their primordial disks, whereas only 6/25 of their higher-mass counterparts in the same cluster showed any evidence for disks.  It is conceivable that the initial disk fraction around brown dwarfs and stars is similar, but brown dwarf disks have longer lifetimes because they accrete more slowly \citep{2007ApJ...657..511A}.  Small sample sizes have made it difficult to say whether brown dwarf disks \emph{really} do last longer than disks around stars.  The reported uncertainties in disk fractions are typically 10$-$20\% \citep{2012MNRAS.420.2497R}.  \citet{2007ApJ...660.1517S} and \citet{2009ApJ...705.1173R} report disk fractions in discord at $>2\sigma$ for distinct subsamples of brown dwarf members of Upper Scorpius.  Larger samples of brown dwarfs, with and without disks will be helpful for settling the role of stellar mass on disk lifetime.  

\subsection{Young brown dwarfs for the study of low mass sources}
Firm members of young star forming clusters are especially valuable because \emph{we know the age} of the clusters, so we can directly test substellar interior and substellar atmosphere models.  There are not enough class III objects against which to test these models across a broad range of surface gravity.  A collection of young, diskless brown dwarfs is needed.  As brown dwarfs monotonically cool and contract, they exhibit M, L, and T spectral types \citep{2012ApJS..201...19D}.  Studies of the solar neighborhoood have succeeded at finding $\sim$Gyr populations of L, T, and Y dwarfs from their large proper motions and colors \citep{2011ApJS..197...19K}.  L-type and T-type objects now number in the hundreds \citep{2010AJ....139.1808S,2011ApJS..197...19K}.  Meanwhile, studies of nearby young moving groups reveal ``juvenile'' brown dwarfs \citep{2013ApJ...772...79A}, with ages $\sim20-100$ Myr.  Juvenile brown dwarfs serve as analogs to directly imaged exoplanets, which exhibit similar spectral types to brown dwarfs \citep{2015ApJ...798L...3C}.  Young brown dwarfs discovered as companions to Herbig AeBe stars are analogous to lower mass planets forming in a disk around lower mass stars.  The youngest brown dwarfs in nearby star forming clusters complete this picture of brown dwarf evolution across the time and mass domain.

\subsection{Mass estimation}

There is a minimum mass below which brown dwarfs cannot form directly from gravitational collapse of a gas cloud.  \citet{1976MNRAS.176..483R} showed that this minimum mass is about 2 $M_{Jup}$.  Though several candidate $<$2 $M_{Jup}$ free floating objects have been found \citep{2015A&A...574A.118P}, only one has withstood spectroscopic follow up \citep{2010ApJ...709L.158M}.  Their low luminosity and intrinsic rarity hinder observations.

We do not yet have dynamical mass measurements of young brown dwarfs in nearby star forming regions.  Instead, the usual strategy for assigning masses to young stars is by placing individual objects on an HR diagram, and comparing their locations to stellar evolutionary model tracks.  There are several flaws with this strategy, especially at low masses.  Specifically, bolometric corrections and effective temperatures are highly uncertain for spectral types later than M5 \citep{2013ApJS..208....9P}, not to mention that the evolutionary model tracks differ by factors of 2 on the mass at a given place on the HR diagram \citep{1994ApJS...90..467D,1998A&A...337..403B}.  There are additional systematic uncertainties that go into assigning an effective temperature based solely on a spectral type.  For example, correcting for chromospheric activity can affect masses by $\sim3-100\%$ \citep{2014ApJ...796..119S}.

\subsection{Previous surveys of young brown dwarfs in nearby star forming regions- Difficulties in identification and biases therefrom}

To understand brown dwarfs, we need a large, complete, unbiased sample.  Barring that, we at least need representative examples of brown dwarf/disk systems.  Detection techniques for brown dwarfs have been ``success-oriented''.  Low luminosity ($\lesssim10^{-3}L_{\odot}$) brown dwarfs are hard to find among copious reddened background stars and galaxies.

There have been dozens, if not hundreds, of searches for brown dwarfs in nearby star forming clusters (see \citet{2012ARA&A..50...65L} for a recent review).  Some representative examples include \cite{2004ApJ...602..816L,2012A&A...539A.151A,2014ApJ...785..159M}.  The approaches in these searches all differ in their details, but the main theme is to construct a color magnitude diagram, and select sources that are brighter and redder than some threshold.  This strategy relies on the fact that there are very few luminous red sources in the foreground, since the star forming regions are relatively close by ($\sim150$ pc).  This strategy works fine for stellar sources, which are much more luminous than the background objects.  Brown dwarfs on the other hand exhibit optical and near-IR colors and magnitudes comparable to reddened distant background objects.  At the same time, the intrinsic number density of brown dwarfs is decreasing, so the problem is like searching for a needle-in-a-haystack.  The contamination becomes so unmanageable that additional selection methods become necessary in order to cull the large number of photometric candidates to a number of candidates that is amenable to follow up spectroscopy.  These additional selection methods vary.  One main strategy is simply to select sources that exhibit mid-IR excess \citep{allers06}.  This choice has the virtue of offering a high likelihood of success- 18/19 of the \citet{allers06} sample were confirmed as bona fide young late-type sources \citep{2011ASPC..448..633G}.  Further, the availability of deep mid-IR imaging from \emph{c2d} (and to a lesser depth \emph{WISE}), makes it feasible to carry out this strategy.  

Selection methods demanding the presence of mid-IR excess cannot discover sources without mid-IR excess.  Evolved disks therefore go undiscovered.  These evolved disks include debris disks and Class III analogs that have already undergone a phase of primodial disk clearing (perhaps by planet formation).  This tautological bias means that brown dwarfs with disks are probably over-represented in our counts of young brown dwarfs.  It is hard to say to what extent the number of diskless brown dwarfs is under-estimated.  It is conceivable that diskless brown dwarfs are rare, because the disk dispersal timescale is longer in brown dwarfs \citep{2008ApJ...681.1584R}.  Or it could be that disks that have already evolved went through a period of vigorous accretion that caused the sources to appear underluminous for their age \citep{2009ApJ...702L..27B}, further hindering their unambiguous selection and detection.  In the latter case, diskless brown dwarfs, and therefore brown dwarfs altogether, would be more common, raising the question- ``What else is there?''.  For example, are there pure photosphere sources or more extreme transition disks?  Are we over-estimating accretion rates by selecting for strong disks?  Are we missing objects hidden behind their disks by viewing angle?  What are the different modes of evolution?  Again, to fully understand the YSO phase of BDs, we need to have examples of the full range of source types. 

%\subsubsection{The range of brown dwarfs in $T_{\mathrm{eff}}$ and character}
%Assuming an age of \verb|xx| Myr, the range of absolute magnitudes of YBDs is expected to be \verb|xx| from models \citep{2002A&A...382..563B}.  Couple the age uncertainty and episodic accretion uncertainty, we get a range of \verb|xx|.  The range of distances to known nearby young clusters is $120 < d \mathrm{(pc)} < 300$.  These translate to apparent magnitudes of \verb|$<J<$| and \verb|$<I<$|, assuming no extinction.  Assuming extinction of \verb|$A_V = 8$| and \verb|$R_V = 3.1$|, the apparent magnitudes become \verb|$<J<$| and \verb|$<I<$|.  \citet{2013ApJS..208....9P} show the range of $T_{\mathrm{eff}}$ and $L_{\mathrm{bol}}$ for young sources.  Here we see a range of \verb|x-y| for the colors of young objects.

\subsection{Motivation for this project}


In order to trace the lifetimes of brown dwarf disks, statistically significant samples must be assembled for clusters of ages 1$-$10 Myr and searched for infrared excesses.  What is lacking is a sample of brown dwarfs that is unbiased by selection techniques based on infrared excess.  This work assembles a sample of young brown dwarfs both with and without mid-infrared excess.

One of the main ideas of this work is that we take a more catholic approach to finding objects but with a sure way to verify them.  We are sensitive to class III analogs.  We investigate the properties of a broader population of young brown dwarfs to provide a better framework for the early evolution of these objects.

We introduce a custom filter centered at 1.4$\;\mu$m, and demonstrate how measurements with this filter can facilitate selection of young brown dwarfs.  We can therefore select hundreds of candidates that show intrinsic photospheric water absorption.  Most of these objects will be old field M dwarfs.  But some of them will be young brown dwarfs.  We therefore use multi-object spectroscopy to identify the rare young objects from their more numerous old field counterparts.

Ancilary information alone, like the $W-$filter, variability, X-rays, or proper motion, cannot establish membership in a young cluster.  Spectroscopy offers a method for assigning an effective temperature, which when combined with the bolometric luminosity, can be used to assign a position on an HR diagram.  The HR diagram can then be used to coarsely assign masses and ages, about which we have already mentioned shortcomings.  Still, the HR diagram is an exceptionally powerful vehicle for understanding the aggregate properties of a cluster, and the coarse properties of individual sources.  Above all the HR diagram lends strong evidence in favor of$-$or against$-$cluster membership of individual objects.  Spectroscopy therefore drives all observational choices, because of its leading role in confirmation.

The lowest mass young brown dwarfs in nearby star forming clusters are at the sensitivity limit of existing astronomical spectrographs.  Futhermore, spectral resolution, spectral bandwidth, and number of targets all compete for the same spectrograph detector real estate.  The wavelength and sensitivity requirements drive the choice of a large optical/near-IR spectrograph on a 6$-$10 meter class telescope.  The instrument needs a wide field-of-view to cover a large enough portion of nearby star forming regions, which occupy degrees on the sky.  The sensitivity and wide-field requirements drive the instrument choice to the largest product of collecting area $A$ and solid angle $\Omega$.  Few existing instruments have a wide-enough $A\Omega$ to be feasible.  The IMACS spectrograph has one of the largest $A\Omega$ products, and offers a multi-object mode in which $\sim100$ sources can be observed in parallel.  The optical wavelength range of IMACS facilitates spectral classification which has historical been performed in the red visual portion of the spectrum.


We chose \emph{Ophiuchus} because it is one of the closest clusters of very young YSOs \citep{2008ApJ...675L..29L}.  We elected to focus on an off-core region since it offers lines of sight of extinction low enough ($A_V\lesssim 10$) to make $I-$band spectroscopy possible, but high enough to ensure an embedded population \citep{2008A&A...489..143L}.  Deep near-IR and optical photometric data exist throughout the wavelength range of interest from \citet{allers06}.  The existing \citet{allers06} photometry includes about 0.7 square degrees of $I-$band through MIPS $[24]$.  The photometric depths are sufficient to detect a 2 $M_{Jup}$ source in all except the IRAC $[5.8]$ and $[8.0]$ bands.  Additionally, we use about 0.5 square degrees of deep \emph{Spitzer} IRAC $[5.8]$ and $[8.0]$ photometry made available by \citet{2010ApJ...720.1374H}, making portions of our catalog sensitive to the photospheres of 2 $M_{\mathrm{Jup}}$ objects.  
This region has been searched previously for young brown dwarfs.  \citet{allers06} used deep near-IR photometry and \emph{c2d} mid-IR photometry to identify 19 candidates with mid-IR excess.  \citet{2011ASPC..448..633G} spectrally classified all but two of these candidates, finding a spectral type range M2$-$L2.5.  \citet{2006AJ....131.3016S} combined $R-$ and $I-$ band photometry with 2MASS, and optical spectroscopy to identify 30 likely new brown dwarf members for the Upper Scorpius OB association (Upper Sco).  Finally, \citet{2010ApJ...720.1374H} leveraged deep \emph{Spitzer} observations to report 18 new candidates selected for their position on color-magnitude diagrams and their evidence for photospheric $W-$band absorption.  In total our chosen region contains 7 sources from \citet{allers06} and 3 sources from \citet{2006AJ....131.3016S} in addition to the candidates from \citet{2010ApJ...720.1374H}.  

Our approach is to apply a broad set of strategies for identifying targets to confirm with spectroscopy.  We select for sources that are bright enough to observe in $I-$band with IMACS on Magellan.  The selection strategies include looking for evidence of disk, intrinsic water absorption, and red optical $-$ near-IR colors.    We place fields to include the largest number of best sources, and other, less promising candidates that the instrument properties allow us to observe along with these sources.  We follow up and confirm a large number of candidates with multi-object $I-$ band spectroscopy.  We characterize and spectral type sources with $I-$band spectroscopy.  We combine several lines of evidence for youth from the spectra, from the mid-IR photometry, and from the position on an HR diagram.  At the end we summarize our sample, and use it to derive insights about the parent population of brown dwarfs.  We can ultimately assess how well the selection methods work.

%------------------------------------------------------------------------------------------
\section{Input catalog and candidate selection}
%------------------------------------------------------------------------------------------
\subsection{Photometry}
Figure \ref{fig_star_chart} shows the survey region, which is centered on---but larger than---the region of \citet{allers06} for the \emph{Ophiuchus} portion of their survey of \emph{Ophiuchus}, \emph{Chamaeleon}, and \emph{Lupus}.  Our photometric survey extends the spatial coverage of the \citet{allers06} survey towards the core of \emph{Ophiuchus}.  Figure \ref{fig_star_chart} shows the 2MASS extinction map \citep{2008A&A...489..143L} from the COMPLETE project \citep{2006AJ....131.2921R} with lighter regions having higher extinctions.  We use the same $I$ and $K$ photometry as \citet{allers06}, and the same $JH$ photometry, except for the additional photometry towards the core, which is reduced identically to \citet{allers06}.  The $JHK$ photometry is from ISPI at CTIO, and is about 3.3 magnitudes deeper than 2MASS \citep{allers06}.  The $JH$ photometry has over twice as much spatial coverage as the $K-$ band photometry.  Figure \ref{fig_star_chart} shows the footprint for the ISPI $JH$ survey region, demarcated by the solid blue border extending towards the \emph{Ophiuchus} core.  The green border shows the footprint for the $K-$band photometry.  The positions of known brown dwarfs are shown as blue diamonds \citep{allers06}, red and green triangles \citep{2013A&A...559A.126A}, and yellow squares \citep{2006AJ....131.3016S}.  Table \ref{tbl_photomCatalog} lists the photometric depths and number of detections in bands $I$, $J$, $W$, $H$, $K$, IRAC $[3.6]$, $[4.5]$, $[5.8]$, $[8.0]$, and MIPS $[24]$.

\begin{figure}[ht!]
\centering
\includegraphics[scale=0.5]{chIMACS/figures/Ophiuchus_map}
\caption[Map of the region of \emph{Ophiuchus} targeted in our survey]{Map of the region targeted by our survey.  The background grayscale image is the 2MASS NICER extinction map \citep{2008A&A...489..143L} from the COMPLETE project \citep{2006AJ....131.2921R}.  The solid line boundaries outline our regions of custom photometry with the following color scheme: \emph{blue solid lines} for the ISPI $JH$ catalog, \emph{green} for ISPI $K-$band, \emph{red} for \emph{c2deep}, \emph{magenta} for $I-$band, and \emph{yellow} for the custom $W-$band.  The large \emph{blue dashed} box defines our 2MASS $JH$ catalog.  The \emph{yellow squares} concentrated toward the right side of the image are from previous surveys of \emph{Upper Scorpius} \cite{2006AJ....131.3016S}.  \emph{Ophiuchus} members are concentrated around the high extinction core to the left of the image.  All known brown dwarfs detected by \emph{Herschel} are shown as \emph{green upward pointing triangles}, while brown dwarfs not detected by \emph{Herschel} are \emph{red downward pointing triangles}, \cite{2013A&A...559A.126A}. \emph{Cyan} circles are from \citet{2010ApJ...720.1374H}, and blue diamonds are from \citet{allers06}.\label{fig_star_chart}}
\end{figure}


\subsubsection{c2deep from Harvey et al.}
We acquired a catalog of deep $[5.8]$ and $[8.0]$ photometry from \citet{2010ApJ...720.1374H}.  This region is shown as the red solid line in Figure \ref{fig_star_chart}.  This region overlaps the region with the most pan-chromatic photometry available ($K$, green; $W$, yellow; $I$, magenta), and extends north of the region in which $JH$ ISPI data is available (blue solid line).  The \citet{2010ApJ...720.1374H} photometry is 1.1$-$2.4 magnitudes deeper than c2d, which enables detection of disks of 2 Jupiter mass objects in Ophiuchus \citep{allers06}.  The cyan circles in Figure \ref{fig_star_chart} are the 18 candidates from \citet{2010ApJ...720.1374H}.  

\subsubsection{$W-$ filter}
Allers et al. (in prep) designed and implemented a custom filter centered at 1.4 $\mu$m, intermediate between the $J-$ and $H-$ atmospheric windows.  This filter is about 0.2 $\mu$m wide, cutting on near the long end of $J$ and cutting off near the short end of $H$.  This photometric band was designed to distinguish between the photospheric emission of late M and L spectral types and that of reddened FGK and early M spectral types.  Cool stars and brown dwarfs with late M spectral types demonstrate intrinsic water absorption in their photospheres.  There is a strong water absorption band centered around 1.4 $\mu$m.  The filter can also distinguish high$-z$ galaxies, which will not demonstrate intrinsic photospheric water absorption as brown dwarfs do.  The broad-band IJHK colors for FGK and early M sources are indistinguishable, while the $W-$filter band photometry shows a deficit in M-type spectra (Allers et al. \emph{in prep}).

One potential pitfall with the $W-$band is that this wavelength region has high and variable telluric water absorption.  Telluric water absorption is problematic for two reasons.  First, the absolute transmission through the filter will be low.  Second, the variability of the telluric water absorption will manifest as shifting wavelength-dependent zero-points. To ameliorate these problems we observed from the high, dry site Mauna Kea, where the telluric water absorption is much lower than other sites.  Ideally the $W-$band photometry should be taken contemporaneously with $J-$ and $H-$ bands, in order to overcome intrinsic source variability, which is especially acute for late type young brown dwarfs that demonstrate accretion variability and star spots \citep{2008A&A...485..155A,2014AJ....147...82C}.

Despite the telluric absorption challenge, the $W-$ filter works- it has already been successfully applied to select brown dwarfs with over 90\% confirmation rate (K. Allers, \emph{priv. comm.}).  The $W-$ filter has been successful at recovering known objects, and has been checked in the \emph{Ophiuchus} core region, IC348, and NGC1333.

One main potential advantage of the $W-$band filter is its \emph{reddening insensitivity}.  Allers et al. (in prep.) derive a $Q-$parameter from $J$, $W$, and $H$ bands, following the form of \citet{2000ApJ...541..977N}.  The $Q$ value is the depth of the $W-$band normalized by the $J-$ and $H-$bands, and can be tuned for particular lines of sight to mitigate the effect of reddening (Allers \emph{et al.}, in prep.).  A flat spectrum source would exhibit $Q=0$, whereas a source with strong photospheric water absorption indicative of a late M spectral type would have a value of $Q$ in the vicinity of $-1$.  \citet{2000ApJ...541..977N} have shown that the $Q$ values scales with spectral types in the range M0$-$M9, for both sources in young clusters and standard stars.  

The $W-$ band photometry in this work was observed with the University of Hawaii 88" telescope with the ULBCAM imager.  The individual photometric frames were dark subtracted, divided by a normalized flat, and median combined following the procedure of \cite{allers06}.  We use the Source Extractor \citep{1996A&AS..117..393B} to perform aperture photometry on the $W-$band photometry.  The yellow outline in Figure \ref{fig_star_chart} shows the region for which $W-$band photometry exists.

%%%%%%%%%%%%%%%%%%%%%%%%%%%%%%%%%%%%%%%%
% TABLE - Catalog Photometry
%%%%%%%%%%%%%%%%%%%%%%%%%%%%%%%%%%%%%%%%
\begin{deluxetable}{ccccc}
\tablecolumns{5}
\tabletypesize{\footnotesize}
\tablecaption{Photometry in this study \label{tbl_photomCatalog}}
\tablewidth{0pt}
\tablehead{
\colhead{Band} &
\colhead{N 3$\sigma$ Detections} &
\colhead{10$\sigma$ mag} &
\colhead{Source} &
\colhead{Ref.} }
\startdata
	\multicolumn{5}{c}{ISPI $JH$ Catalog} \\
	\hline
	$I$  & 34467 & 23.0 & MOSAIC/CTIO 4m &  1 \\
	$J$  & 54373 & 19.4 & ISPI/CTIO &  1 \\
	$W$  & 31205 & 19.1 & ULBCAM/U. Hawaii 88" &  1 \\
	$H$  & 54373 & 18.8 & ISPI/CTIO &  1 \\
	$K$  & 20470 & 18.0 & ISPI/CTIO &  1 \\
	$[5.8] $  & 9135 & 15.7 & \emph{c2deep} &  2 \\
	$[8.0]$ & 1948 & 14.9 & \emph{c2deep} &  2 \\
	$[3.6]$  & 37423 & 16.1 & c2d &  3 \\
	$[4.5]$  & 30345 & 15.5 & c2d &  3 \\
	$[5.8]$  & 11542 & 15.7 & c2d &  3 \\
	$[8.0]$  & 2878 & 12.5 & c2d &  3 \\
	$[24]$\tablenotemark{a}  & 145 & $\sim7.9$& c2d &  3 \\
	\hline
	\multicolumn{5}{c}{2MASS $JH$ Catalog} \\
	\hline
	$I$  & 6939 & 22.9 & MOSAIC/CTIO 4m &  1 \\
	$J$  & 27208 & 16.1 & 2MASS &  4 \\
	$W$  & 11226 & 16.3 & ULBCAM/U. Hawaii 88" &  1 \\
	$H$  & 27618 & 15.3 & 2MASS &  4 \\
	$K$  & 25254 & 14.5 & 2MASS &  4 \\
	$[5.8]$  & 4891 & 15.5 & \emph{c2deep} &  2 \\
	$[8.0]$ & 2173 & 14.8 & \emph{c2deep} &  2 \\
	$[3.6]$  & 18716 & 15.3 & c2d &  3 \\
	$[4.5]$  & 18604 & 15.4 & c2d &  3 \\
	$[5.8]$  & 12859 & 13.2 & c2d &  3 \\
	$[8.0]$  & 5973 & 12.4 & c2d &  3 \\
	$[24]$\tablenotemark{a}  & 452 & $\sim7.9$& c2d &  3 \\
	
\enddata
\tablenotetext{a}{The reported depth is for 8~$\sigma$.}

\tablerefs{
(1)~\citet{allers06}
(2)~\cite{2010ApJ...720.1374H}
(3)~c2d, \cite{2009ApJS..181..321E}
(4)~2MASS, \cite{2006AJ....131.1163S}
}

\end{deluxetable}
%%%%%%%%%%%%%%%%%%%%%%%%%%%%%%%%%%%%%%%%

\subsection{ISPI $JH$ Catalog ($\sim3.3$ magnitudes deeper than 2MASS)}
We established a source catalog based on the ISPI $JH$ photometry, which had the most spatial coverage.  Membership to the catalog requires $5\; \sigma$ detection in \emph{both} $J$ and $H$ bands.  \emph{A total of 54373 sources met this criterion}.  This catalog is demarcated by the solid blue line in Figure \ref{fig_star_chart}.  We refer to this catalog as the ``ISPI $JH$ Catalog''.  We justified the $JH$ detection requirement because these are the bands in which young brown dwarfs peak in flux.   Sources below our 5~$\sigma$ detection limits in $J-$ and $H-$ bands would stand little chance of follow up near-IR spectroscopy with existing instrumentation.  Further, sources with red optical minus IR colors---an expectation for late-type reddened young brown dwarfs---would stand even less of a chance of optical spectroscopic follow up.  

The astrometric solution for the catalog comes from the ISPI astrometry, which is pegged to 2MASS (see \citet{allers06} for details).

We cross-match the 54373 $JH$ selected source catalog with each of the other photometric catalogs of $K$, $W$, $[3.6]$, $[4.5]$, $[5.8]$ from c2d, $[5.8]$ from c2deep, $[8.0]$ from c2d, $[8.0]$ from c2deep, and MIPS $[24]$. In cross matching, any source within a small angular separation of a $JH$ catalog member is matched to the $JH$ member, with angular separations typically between 0.5 and 1.5" depending on the quality of the WCS solution in the imaging frames.  We do not attempt to calculate an upper limit for undetected sources.

Table \ref{tbl_photomCatalog} lists the number of $>3\sigma$ detections in each band, and the typical 10 $\sigma$ magnitude, which was computed as the median flux of sources with signal noise ratio between 9.5 and 10.7 $\sigma$.

\subsection{2MASS $JH$ Catalog}
We also made a shallow catalog constructed with 2MASS $JH$ photometry instead of the deeper ISPI $JH$ photometry.  This catalog had the virtue that its spatial coverage included all of the auxiliary imaging bands that extended past the footprint of the ISPI imaging.  In particular, several candidates reported in \citet{2010ApJ...720.1374H} did not appear in the ISPI $JH$ Catalog since they were outside the ISPI imaging region, but inside the c2deep imaging region.

We selected all sources in 2MASS with in a region with right ascension between $16^h19^m30^s$ and $16^h26^m$, and declinations between $-25^{\circ}$ and $-23^{\circ}$ (blue dashed line in Figure \ref{fig_star_chart}).  This selection resulted in 28092 sources.  We followed the same procedure as we did above for performing catalog matching.  Table \ref{tbl_photomCatalog} lists the properties for the 2MASS $JH$ catalog.


\subsection{Candidate selection}
%Describe in detail the wittle down of the sample: Total number of sources, number used (i.e. detect in I band or exceptions from extra region)  We need to diagram this section.  Then how slit choices were made- giving numbers of sources at each stage.

%Now we have N spectra.  Ditch M of these as too faint or clobbered.  Now have Q left over.  How do we characterize these for Teff, Halpha, sodium index.

Equipped with our photometric catalogs, we were prepared to select candidates for spectroscopic follow up.  The IMACS spectrograph on Magellan can measure hundreds of sources, but not tens of thousands.  We had to reduce the number of follow up sources from 54373 (deep $JH$) and 28092 (shallow $JH$) to about 300 candidates, a down-select factor of about 0.5\%.  

We first considered the sensitivity of the instrument to set the limiting magnitude for $I-$ band spectroscopic follow up.  Our 10 $\sigma$ photometric depth in $I-$ band is 23.0.  We set a limit of $I=21.6$ for follow up spectroscopy.  We estimated that this threshold could provide sufficient signal-to-noise ratio for at least coarse spectral classification- ``Is the source late type or not?'', though higher signal to noise or auxiliary information would be needed to assign youth and therefore membership to the source.  Only two thirds of catalog had $I-$band measurements available, due to the smaller spatial coverage of the $I-$band imaging.  Of those sources with $I-$band detections, only 38\% had $I>21.6$.  We assigned a proxy limiting magnitude of $J>17.4$ for sources undetected at $I-$band.


We devised four physically motivated selection criteria.  Each criterion incorporated a different aspect of known or expected physics.  The criteria are broadly $IJHK$ colors, $H- [3.6]$, custom $W-$filter, and mid-IR excess.  The criteria are binary-- the candidate met the criterion or it did not.  In this section we describe the motivation and justification for the details of these four selection criteria.  In the next subsection we describe the method we used for combining the various combinations of these 4 criteria to issue an observation priority for each candidate.  Table \ref{tbl_selectionScores} summarizes the four criteria.

%%%%%%%%%%%%%%%%%%%%%%%%%%%%%%%%%%%%%%%%
% TABLE - Scores
%%%%%%%%%%%%%%%%%%%%%%%%%%%%%%%%%%%%%%%%
\begin{deluxetable}{ccccccc}
\tablecolumns{7} 
\tabletypesize{\small}
\tablecaption{Binary selection criteria \label{tbl_selectionScores}}
\tablewidth{0pt}
\tablehead{
\colhead{Score} &
\colhead{Criterion} &
\colhead{IDs} &
\colhead{$N$ meet} &
\colhead{$N$ eligible} &
\colhead{$\%$ catalog} &
\colhead{$\%$ eligible} \\}
\startdata
	8 & $IJHK$ & NIR & 3680 & 34469 & 6.8 & 10.7\\
	4 & $H-[3.6]$, $K-[3.6]$ & $[3.6]$ & 23112 & 42268 & 43.5 & 54.6 \\
	2 & $W-$ band & $W$ & 1180 & 31348 & 2.2 & 3.8\\
	1 & Mid-IR excess & Disk & 1407 & 14095 & 2.6 & 10.0\\
\enddata
\tablecomments{The number of catalog members eligible to meet a criterion differs from the length of the catalog (54373) due to missing data in the form of non-detections or non-observation.}
\end{deluxetable}

\subsubsection{Near-IR photometric selection}
\label{sec_NIR_selection}

For those sources where we had $I-$ band available, we set a selection criterion based on optical minus near-IR.  The main point of this criterion was to select sources with colors indicative of---or at least \emph{consistent} with---late M spectral types.  Known young brown dwarfs at 1 Myr have spectral types later than about M6 \citep{2012ARA&A..50...65L}.  The full range of colors for young brown dwarfs is not known.  Our approach was to compile examples of colors of known young brown dwarfs and design criteria that would recover these objects.  The final criterion is composed of three color cuts, and one magnitude cut.

The color cuts rely heavily on $I-$band photometry, owing to this band's discriminatory power of separating red optical colors indicative of young brown dwarfs from bluer optical colors indicative of unreddened or minimally reddened stars.  The lack of $I-$band measurements for some sources was because of the limited spatial coverage of the $I-$band mosaics, not because of sensitivity limits (our $I-$ band photometry is exceptionally deep).  Because of our heavy reliance on $I-$band, sources without $I-$band photometry were ineligible to meet the near-IR photometric selection criterion.  A total of 34469 out of 54373 sources had $I-$band measurements.  This cut reduced the number of sources from 100\% to 63.4\%.

As mentioned above, one of the clearest signatures of brown dwarfs is their red optical color, since optical  ($R-,I-, z-$ band; $\lambda \sim  0.8 \; \mu$m) is short of the peak transmission of the cool ($T(K) <$ 2900; $\lambda \sim$ 1.0-2.0 $\mu$m) substellar photospheres of brown dwarfs.  However the large optical minus infrared color is a hallmark of both brown dwarfs and high-$z$ galaxies, and can be mimicked by the effect of high reddening.  The $I-J$ color of brown dwarfs with spectral types later than $\sim$M6 is typically greater than $\sim2.85$ \citep{allers06}.  Reddening would only act to make this value larger.  So we required $I-J +\sigma_{I-J} > 2.85$, where $\sigma_{I-J}$ is the measurement uncertainty of $I-$ and $J-$ bands added in quadrature.  The measurement uncertainty in $I-$ and $J-$ bands are the statistical uncertainties in the measurement and do not include calibration uncertainties.  The statistical uncertainty is evaluated for each source, and ranges from 0.001 to 0.1 for $H-$, and $I-$ bands.

The effect of including the uncertainty in this color cut is to include sources that are consistent with---albeit with values nominally below---our threshold.  For example, a hypothetical candidate with $I-J=$2.80 and combined measurement uncertainties in $I-$ and $J-$ bands $\sigma_{I-J} = \sqrt{\sigma^2_{I}+\sigma^2_{J}}=$0.07 would have $I-J + \sigma_{I-J} = 2.87$ and would therefore exceed our 2.85 threshold.  This inclusive selection strategy is perceptible in Figure \ref{fig_NIR_selection_JH}.  The motivation for setting an inclusive boundary was to avoid cutting out interesting albeit low signal-to-noise ratio sources at the margin.

The $I-J$ color cut was met by 12548/54373 ($23.1\%$) of the total catalog members, or 12548/34469 (35.4\%) of those sources possessing $I-$band measurements

Our next color cut is to select for blue intrinsic $J-H$ colors.  The motivation for this color cut is that observed young brown dwarfs demonstrate a tight range in intrinsic $J-H$ (0.58-0.71 for M5$-$M9 \citep{2010ApJS..186..111L}), so we could eliminate contamination by setting the threshold redward of the largest $J-H$ we could expect.  We set that threshold at 0.84156, on the MKO scale\footnote{When performing candidate selection on the 2MASS $JH$ Catalog, we altered our color-cut thresholds to the 2MASS photometric system.}.  A large extinction could pull the observed value past the 0.84156 threshold.  So we constructed a reddening line that related the observed $I-J$ with the observed $J-H$ through the selective extinction ratios $A(\lambda)/A_V$ compiled from the Asiago Database of Photometric Systems (ADPS) \citep{2000A&AS..147..361M} with $R_V=3.1$.  The color cut is: $I-J + \sigma_{I-J} > 2.85 + [(J-H) - 0.84156 + \sigma_{J-H}]\frac{A_I-A_J}{A_J - A_H}$.  Where $\sigma_{I-J}$ is the uncertainty in the $I-$ and $J-$ bands added in quadrature, and $\sigma_{J-H}$ is the photmetric uncertainty in $J-$ and $H-$ bands added in quadrature, using the same methodology as already discussed.  The ratio $\frac{A_I-A_J}{A_J - A_H}$ is equal to 3.33, and takes into account the selective extinction.  Figure \ref{fig_NIR_selection_JH} shows this color cut as the diagonal line.  The upper left portion of the plot meets both color cuts discussed so far.  The upper right corner has red $I-J$ color, but has $J-H$ color that is too red for an M9 (which has the largest $J-H$ for any M spectral type) to be explained by reddening alone.  This color cut was met by 8105/54373 sources (14.9\%).  At this point, 3967 sources (7.3\% of the catalog) met all the previous color cuts.

\begin{figure}[ht!]
\centering
\includegraphics[scale=0.6]{chIMACS/figures/NIR_selection_allers_M6_L2}
\caption[Photometric selection color-color cuts in $I-J$ vs $J-H$]{Color-color cuts in $I-J$ vs $J-H$. The horizontal dashed line shows $I-J=2.85$.  The tilted dotted line demonstrates the selection for intrinsic blue $J-H$ colors ($<0.84156$), allowing for reddening.  The grayscale Hess diagram in the background shows the density of points in each color bin, constructed from 34469 sources with $I-$band detections.  The points in the upper left quadrant of the diagram meet the color cuts 2 and 3 from the list in Section \ref{sec_NIR_selection}.  Sources that meet the color cut are illustrated with black circles with white outlines.  Some of the sources that meet the selection lie outside the thresholds shown as the red dotted and red dashed lines.  The magenta circles are sources from \citet{allers06} that have spectral types between M6 and L2 \citep{2011ASPC..448..633G}. \label{fig_NIR_selection_JH}}
\end{figure}

We leveraged the $K-$band photometry when it was available to make another color cut.  Like the $I-$band data, the $K-$ band had incomplete spatial coverage-- only 38\% of sources had a $K-$band measurement available.  Our strategy here was similar to the color cut above selecting for blue intrinsic $J-H$ color, except now we are selecting for intrinsic blue $J-K$ color.  We assumed a maximum $J-K$ for brown dwarfs of 1.37662 in the MKO system.  We then built a reddening relationship between $I-J$ and $J-K$ that takes into account the selective extinction from ADPS, as above.  Lastly, we \emph{removed} sources that were redder than this curve, removing only sources that had $K-$band available.  The sources we removed satisfied this criterion:  $I-J + \sigma_{I-J} < 2.85 + [(J-K) - 1.37662 - \sigma_{J-K}]\frac{A_I-A_J}{A_J - A_K}$, where $\sigma_{I-J}$ is the same as it is above, and $\sigma_{J-K}$ is the photometric measurement uncertainty in $J-$ and $K-$ bands added in quadrature.  We removed 11020/54373 sources, which means that the complement, 43353 passed on.  Combining this criterion with the previous color cuts brings the passage rate from 7.3\% from the previous cuts to 6.8\%.

\begin{figure}[ht!]
\centering
\includegraphics[scale=0.6]{chIMACS/figures/NIR_selection_JK_allers_M6_L2}
\caption[Photometric selection color-color cuts in $I-J$ vs $J-K$]{Color-color cuts in $I-J$ vs $J-K$. The horizontal dashed line shows $I-J=2.85$.  The tilted dotted line demonstrates the selection for intrinsic blue $J-K$ colors ($<1.37662$), allowing for reddening.  The grayscale Hess diagram in the background shows the density of points in each color bin, constructed from 19733 sources with both $I-$ and $K-$band detections.  The points in the upper left quadrant of the diagram meet the color cuts 2 and 4 from the list in Section \ref{sec_NIR_selection}.  The square boxes plotted are sources from \citet{allers06} which have spectral types between M6 and L2 \citep{2011ASPC..448..633G}.\label{fig_NIR_selection_JK}}
\end{figure}

Lastly, we applied a $K-$ magnitude cut, with the intention of finding the faintest and therefore lowest luminosity young brown dwarfs.  For a given cluster age, the lowest luminosity young brown dwarfs are also likely to be the lowest mass.  We deemed these lowest mass sources to be most scientifically interesting.  Following \citet{allers06}, we assumed an absolute $K-$band magnitude of 8.01 for a young M9, and a distance modulus $\mu=5.48$, for a $K-$band apparent magnitude cut of $K + \sigma_{K} > 13.49$, where $\sigma_K$ is the photometric measurement uncertainty in the $K-$band.  Since $K-$band data was unavailable for many sources, we also removed sources consistent with $H-$band apparent magnitudes less than 8.51 plus the distance modulus, keeping sources with: $H+\sigma_{H} > 13.99$.  Figure \ref{fig_NIR_selection_H} shows the number density Hess diagram of all 54373 sources in the $H-$ versus $J-H$ diagram.  The large magenta circles are sources from \citet{allers06} towards \emph{Ophiuchus}.  The only \citet{allers06} sources fainter than our $H>13.99$ threshold are the latest type sources known in that region.  Again, we sought sources fainter than this limit since we were most interested in low luminosity, low mass brown dwarfs.  This criterion is likely to cut sources with spectral types earlier than M9, as demonstrated in Figure \ref{fig_NIR_selection_H}.  We do not plot the $K-$magitude cut plot, but it looks similar to Figure \ref{fig_NIR_selection_H}.

This magnitude cut removed 1105 sources out of the 54373 source catalog.  When combined with the previous color cuts above, this magnitude cut removed 14 sources: OPH\_675, OPH\_672, OPH\_679, OPH\_678, OPH\_673, OPH\_674, OPH\_12081, OPH\_12079, OPH\_3054, OPH\_5413, OPH\_12961, OPH\_13049, OPH\_19613, and OPH\_956.

\begin{figure}[ht!]
\centering
\includegraphics[scale=0.6]{chIMACS/figures/NIR_selection_H_allers_oph}
\caption[$H-$band magnitude selection criterion]{$H-$band Magnitude cut.  The horizontal dashed line shows $H=13.99$.  The grayscale Hess diagram in the background shows the density of points in each color-magnitude bin, constructed from the entire 54373 point ISPI deep $JH$ source catalog.  The circles are sources from \citet{allers06} towards \emph{Ophiuchus}.  The two \citet{allers06} sources with $H>13.99$ have M8 spectral types, while the others are M6 or earlier \citep{2011ASPC..448..633G}. \label{fig_NIR_selection_H}}
\end{figure}


\subsubsection{Mid-IR excess (disk) selection criterion}
\label{sec_midIR_selection}
Sources detected at 24 $\mu$m with greater than 3 $\sigma$ confidence were automatically considered disk sources, since we assumed $[24]$ could not detect photospheric emission for our low luminosity sources.  The 3 $\sigma$ depth for 24 $\mu$m is $\sim~$9.8 magnitudes, so given the assumed distance modulus to \emph{Ophiuchus} of 5.48, the absolute magnitude of the photosphere would have to be 4.3, which is brighter than we expect for young brown dwarfs without disks.  Only 200 sources had $>3\;\sigma$ detection at $[24]$.

We required a $2\;\sigma$ detection in $[3.6]$ and a $2\;\sigma$ detection in any one of $[5.8]$, $[8.0]$, or $[24]$.  Only 14095 sources met this cut.  

Sources had to demonstrate a 2 $\sigma$  excess above both the photospheric $[3.6]-[5.8]$ color and $[3.6]-[8.0]$ color.  We defined the photospheric mid-IR colors of $[3.6]-[5.8]=0.08$ and $[3.6]-[8.0]=0.21$ \citep{2006ApJ...651..502P}.  Figure \ref{fig_midIR_selection} shows this color cut.  In total, 1558 sources met this color cut.  

Note that the mid-IR excess threshold as defined above is fairly liberal- sources marginally above the assumed photospheric emission threshold would be counted as mid-IR excess.  This minuscule degree of excess could be consistent with no disk whatsoever and unusually red mid-IR colors.  Our choice to set this liberal threshold was that the deep \emph{Spitzer} imaging could detect exceptionally weak mid-IR excess.  We would expect diskless young sources to be clustered around the intersection of the dashed and dotted lines in Figure \ref{fig_midIR_selection}.  \emph{Anemic} disks would occupy the region of color-color space between the diskless sources and the colored circles representing the \citet{allers06} sources.  Altogether 1407 sources met all of the above criteria consistent with mid-IR excess.


\begin{figure}[ht!]
\centering
\includegraphics[scale=0.6]{chIMACS/figures/disk_selection_allers}
\caption[Mid-infrared color selection]{$[3.6]-[5.8]$ vs $[3.6]-[8.0]$ color plot, demonstrating the mid-IR excess selection criterion.  The colored 2D histogram (Hess diagram) shows the density of 14095 sources in our sample possessing a 2 $\sigma$ detection in $[3.6]$ and a 2$\sigma$ detection in at least one of $[5.8]$, $[8.0]$, or $[24]$.  The horizontal dashed line is $[3.6]-[8.0]=0.21$, the vertical line is $[3.6]-[8.0]=0.08$ \citep{2006ApJ...651..502P}.  Sources meeting the criteria described in Section \ref{sec_midIR_selection} are primarily in the upper right quadrant.  Some sources with 3$\sigma$ $[24]$ detections in the other quadrants met our disk selection criteria based solely on their $[24]$ detection.  The large colored circles are all sources from \citet{allers06}. \label{fig_midIR_selection}}
\end{figure}

\subsubsection{$W-$ filter selection criterion}
We used $W-$filter photometry to construct a selection for late-type objects.  The extent of water absorption depends on the effective temperature $T_{\mathrm{eff}}$.  We  quantified the degree of intrinsic water absorption with a $Q$ parameter from the trio of $J-$, $W-$, and $H-$ magnitudes (Allers et al. \emph{in prep.}).  The $Q$ parameter is zero for no evidence of water absorption, and negative when water absorption is present.  The water absorption is stronger for cooler photospheres- M6 has a $Q$ value of about -0.5, and L0 has a $Q$ value of about -1.0.  

A total of 31348 sources (57.6\% of the catalog) had $W-$band photometry.  Since all sources in our catalog had $J-$ and $H-$ band photometry, all 31348 sources with a $W-$band measurement were issued a $Q$ value.  Figure \ref{fig_Q_value_histogram} shows the histogram of the $Q$ values.

We designed our $W-$band selection criterion around the value of the $Q$ parameter. Sources met the $W-$band criterion if they were within the boundary $-0.5<Q<-2.5$.  Based on preliminary trials of the $W-$ filter, we expected this range in $Q$ values to correspond to roughly M6 to early L spectral types (Allers et al. \emph{in prep.}).  There were few sources with measurements of $Q<-2.5$, a number so low that we did not expect it to originate from photospheric absorption alone.  Most of these were extremely low $Q-$value sources were attributable to noise.  We required a 2 $\sigma$ measurement of $Q$ inside those bounds, where $\sigma$ is the uncertainty in the $Q$ value propagated from the photometric and calibration uncertainties in the $W-$band.  A total of 1180 sources met the $W-$filter selection criterion.  The number 1180 is 2.1\% of the entire catalog, or 3.7\% of those sources possessing a $W-$band detection.  Figure \ref{fig_Q_value_histogram} shows the selection boundaries as vertical lines in the histogram, with the fraction of sources meeting the criterion highlighted in a different shade.  

\begin{figure}[ht!]
\centering
\includegraphics[scale=0.6]{chIMACS/figures/W_phot_sel_mgs}
\caption[$W-$filter selection method]{Histogram of the $Q$ value derived from $W-$band photometry.  The vertical dashed lines represent the boundaries of the selection method.  The dark colored region region interior to the boundaries indicates the 4\% of sources that were both detected in $W-$band and met the selection criteria.  The measured $Q-$values had to be at least 2$\;\sigma$ within the boundaries on either side, excluding some low-signal to noise ratio sources with nominal values between the boundaries and causing the blue shaded histogram to be below the light shaded parent population histogram.  \label{fig_Q_value_histogram}}
\end{figure}

\subsubsection{$H-[3.6]$ color selection criterion}
We required the $H-[3.6]$ color to appear redward of the assumed photospheric colors.  We assumed photospheric colors from \citet{2006ApJ...651..502P}.  The motivation for this color cut was that extinction and mid-IR excess emission both act to increase $H-[3.6]$.  In principle $[3.6]$ can exhibit finite disk excess, as is seen in $T-Tauri$ stars.  However, we expect brown dwarfs to exhibit negligible excess at $[3.6]$, based on radiative transfer models that show brown dwarf mid-IR excess detaches from the photosphere at wavelengths $>4$ micron \citep{2009MNRAS.394L.141E}.  Our selection criterion required a $[3.6]$ measurement, albeit at any signal-to-noise ratio.  We required $H-[3.6] + \sigma > 0.8$, where $\sigma$ is the photometric measurement uncertainty in $H-$ and $[3.6]$ bands added in quadrature.  For objects with existing $K-$ band photometry we further required $K-[3.6] - \sigma > 0.49$, where $\sigma$ is the photometric measurement uncertainty in $K-$ and $[3.6]$ bands added in quadrature.  All combined, 23112/54373 sources passed this criterion.  Note that we did not apply a signal-to-noise ratio cut on $[3.6]$, which is evinced in Figure \ref{fig_midIR_phot_sel}- sources with large measurement uncertainty still meet our criterion.  Notice also that the $K-[3.6]$ color cut is more stringent than the $H-[3.6]$ color cut; the $K-[3.6]$ demands a 1$\sigma$ excess over 0.49, whereas the $H-[3.6]$ can be merely consistent with 0.8.

\begin{figure}[ht!]
\centering
\includegraphics[scale=0.6]{chIMACS/figures/midIR_phot_sel}
\caption[Mid-infrared photospheric emission selection]{$H-[3.6]$ and $K-[3.6]$ color cuts that comprise the ``mid-IR photosphere'' criterion.  \emph{Left panel-} The gray histrograms show the distribution of the 42268 sources in the catalog with a $[3.6]$ measurement, while the colored histogram shows the distribution of sources with $H-[3.6]$ \emph{consistent with} 0.8 (dashed vertical line) or greater.  The right panel shows an additional and more restrictive color cut applied only to the 18420 sources detected in both $K-$ band and $[3.6]$: these sources had to have $K-[3.6]$ exceed 0.49 (vertical dashed line) by 1 $\sigma$.  The final ``mid-IR photosphere criterion'' is the union of the color cuts in the left and right panels if $K-$band exists, otherwise it is just the sources satisfying the left panel.  \label{fig_midIR_phot_sel}}
\end{figure}

Table \ref{tbl_selectionScores} summarizes the four selection criteria and assigns the criteria abbreviated identifications (NIR, $[3.6]$, $W$, and Disk) that will be used throughout the rest of the text.  The Table lists the number of sources that met each criterion, and the number of sources that were eligible to meet the criterion based on the availability of photometry.  The last two columns list the fraction of the 54373-member source catalog meeting the criterion, and the fraction of sources that were eligible to meet the criterion, and did meet it.

\subsection{Ranking the combinations of selection criteria}

We carried out the $I-$band spectroscopy with IMACS \citep{2011PASP..123..288D} on the \emph{Baade} 6.5 meter Magellan Telescope \citep{2003SPIE.4837..910S} at the Las Campanas Observatory.  IMACS is an imager and multi-object specrograph that uses high-precision laser-cut slit masks to acquire optical spectroscopy over a 27.5 arcminute field of view.  Six IMACS fields cover our photometric region (Figure \ref{fig_star_chart}), but we can only select $\lesssim100$ slits per field.  We therefore need to assign slits to sources based on a combination of our selection cuts and non-interference of spectra on the detector.  

Table \ref{tbl_scoring2Priority} shows our scoring system for translating combinations of our four selection criteria into priority for observation.  There are 16 unique combinations of subsets of selection criteria; sources could meet all four criteria, three criteria (4 combinations), two criteria (6 combinations), only one criterion (4 combinations), or no criteria.  The column \emph{Scores} in Table \ref{tbl_scoring2Priority} lists the combinations of criteria for 10 different combinations of selection methods.  The rank ordering of these combinations is the \emph{Priority} column with 1 being the highest priorty for observation, and 10 being the lowest priority for observation.  The \emph{Net Score} column is the sum of the individual score values, and distinguishes the 16 possible combinations.  Column \emph{IDs} repeats the identifiers associated with each score, \emph{vis-\`a-vis} Table \ref{tbl_selectionScores}.  The last two columns list the number of sources satisfying the combination of selection criteria.  

For example, consider priority 2 sources.  These sources satisfied the NIR, $[3.6]$, and $W$ criteria (scores 8, 4, and 2, respectively), but did not satisfy our disk selection criterion (score 1).  The net score is 14.  Twenty-seven sources satisfy this combination of selection criteria.  We interpret these sources as diskless young brown dwarf candidates.  Since they lack mid-IR excess indicative of a disk, they could also be foreground dwarfs or background dwarfs or giants.  The higher possibility of contamination in this priority compared to priority 1 leads us to assign a lower rank than priority 1.  We rank them above all the rest of the priority groups because the prospect of discovering diskless young brown dwarfs directly addressed our science question of whether or not disks around young brown dwarfs are longer lived than their \emph{T-Tauri} star counterparts \citep{2008ApJ...681.1584R}.  

For another example, consider priority 3 sources.  These sources satisfied the NIR, $[3.6]$, and disk criteria (scores 8, 4, and 1, respectively), but did not satisfy our $W-$band selection criterion (score 2).  The net score is 13.  Seventeen sources satisfy this combination of selection criteria.  Since these sources do not satisfy the $W-$ filter selection method, they are ranked lower than those sources that meet all four selection methods (priority 1).  They are also ranked lower than sources that meet the $W-$ filter selection criterion, but not the disk selection criterion.

In total, 483 sources had a top 10 observing priority.  The other 6 combinations of criteria (net scores 0, 1, 4, 5, 8, and 12) were assigned priority 100, ten times lower priority than our priority 10 sources.  These are listed in Table \ref{tbl_scoring2Priority} as ``other''.

%%%%%%%%%%%%%%%%%%%%%%%%%%%%%%%%%%%%%%%%
% TABLE - Scoring to Priority conversion
%%%%%%%%%%%%%%%%%%%%%%%%%%%%%%%%%%%%%%%%
\begin{deluxetable}{cccccc}
\tablecolumns{6} 
%\tabletypesize{\footnotesize}
\tablecaption{Selection-priority scoring system\label{tbl_scoring2Priority} } 
\tablewidth{0pt}
\tablehead{
\colhead{Priority} &
\colhead{Scores} &
\colhead{Net Score} &
\colhead{IDs} & 
\colhead{$N$ $I<21.6$} &
\colhead{$N$ $I>21.6$} }
\startdata
1	 & 1,2,4,8 &15 & Disk, $W$, [3.6], NIR & 7 & 5 \\
2	 & 2,4,8 &14 & $W$, [3.6], NIR & 27 & 138 \\
3	 & 1,4,8 & 13 & Disk, [3.6], NIR & 17 & 30 \\
4	 & 1,2,8 & 11 & Disk, $W$, NIR & 1 & 0 \\
5	 & 1,2,4 & 7 & Disk, $W$, [3.6] & 15 & 12 \\
6	 & 1,2 & 3 & Disk, $W$ & 10 & 5 \\
7	 & 2,8 & 10 & $W$, NIR & 11 & 109 \\
8	 & 1,8 & 9 & Disk, NIR & 10 & 20 \\
9	 & 2,4 & 6 & $W$, [3.6] & 103 & 158 \\
10 & 2 & 2 & $W$ & 282 & 297 \\
other & - & - & - & \multirow{2}{*}{53116} \\
\enddata
\end{deluxetable}

\subsection{Whittling down from 54373 to $\sim$500}
Equipped with a prioritized list of candidates, we designed multi-object slit masks for IMACS.  The MaskGen software\footnote{See \url{ http://code.obs.carnegiescience.edu/maskgen}} takes a prioritized list of sources and optimizes the placement of slits for the highest number of high priority sources.  Our science goal required a wavelength range of $\sim 650-900$ nm for broad-band spectral typing and H$\alpha$ equivalent width measurement.  Measurement of spectrally narrow gravity-sensitive features indicative of youth \citep{1999ApJ...525..466L,2007AJ....134.2398C,2009AJ....137.3345C} was a top level requirement.  This requirement imposed spectral resolution $R\sim1000$.

Availability of bright stars needed by the telescope active optics system for real-time primary mirror control constrained the available center positions and rotations of the masks.  We settled on center positions that included the most priority 1's, 2's, and 3's.  Table \ref{tbl_IMACSslitMasks} lists the slit masks used in this study.

\begin{landscape}
	%%%%%%%%%%%%%%%%%%%%%%%%%%%%%%%%%%%%%%%%
% TABLE - Slit mask properties
%%%%%%%%%%%%%%%%%%%%%%%%%%%%%%%%%%%%%%%%
\begin{deluxetable}{cccccc}
\tablecolumns{6} 
\tablecaption{IMACS $F/2$ multi-object slit masks \label{tbl_IMACSslitMasks}}
\tablewidth{0pt}
\tablehead{
\colhead{Name} &
\colhead{$N$ slits} &
\colhead{Obs. Dates} &
\colhead{Net Exp. time} &
\colhead{Filter} & 
\colhead{Center RA \& Dec.} \\
\colhead{-} &
\colhead{\#} &
\colhead{YYYYMMDD} &
\colhead{hours} &
\colhead{-} &
\colhead{J2000}}
\startdata
	nc200\_75 & 76 &  20100617 & 2.50  & spectroscopic-2 & 16:21:25 -23:26:42.5\\
	southwes & 72 & 20100617 & 2.00  & spectrosopic-2 & 16:21:44.8 -24:08:45.5\\
	southeas & 45 & 20110707 & 2.25  & 5694-9819 & 16:22:36 -23:54:27 \\
	nc\_2011 & 104 & 20110707 & 4.66 &  5694-9819 & 16:21:25 -23:26:42.5 \\
	sw\_2011 & 101 & 20120513 & 3.58  & WB6300-9500 & 16:21:44.8 -24:08:45.5 \\
	ophwes & 25 & 20120513 & 0.76  & WB6300-9500 & 16:21:29.3 -23:48:38.2 \\
	ophne & 107 & 20120513, 20120514 &  4.38 & WB6300-9500 & 16:21:58.1 -23:17:54.7\\
\enddata
%\tablecomments{The spectroscopic-2 filter has $>80$\% transmission from 3700 $\angstrom$ to 10000 $\angstrom$. The 5694-9819 filter has wavelength range 5694$-$9819 $\angstrom$.  The WB6300-9500 filter has wavelength range 6300$-$9500 $\angstrom$. }
\end{deluxetable}
\end{landscape}

%------------------------------------------------------------------------------------------
\section{Spectroscopic observations}
%------------------------------------------------------------------------------------------

\subsection{IMACS observations}
\subsubsection{Observing Dates and conditions}
We acquired 530 spectra on six different slitmasks over three years.  Table \ref{tbl_IMACSslitMasks} lists the observation dates and on-sky exposure time for all six of the masks.

\subsubsection{Instrumental setup- special note on filters}
We computed spectral response corrections by observing A0V stars and White Dwarfs.  We normalized the spectra for the observed White Dwarfs LTT7987 and LTT3218 by their counterpart model spectra from \citet{2014A&A...568A...9M}.  The A0V stars were normalized by a synthetic model.  The normalized spectral response correction is dominated by a smoothly varying component (filters, grating blaze, etc) and sharp features (telluric atmosphere).

We used three different filters- ``Spectroscopic 2'', ``5694-9819'', and ``WB6300-9500'' (Table \ref{tbl_IMACSslitMasks}).  The ``Spectroscopic 2'' filter had a short wavelength cutoff $<4000 \angstrom$, which resulted in second order overlap for wavelengths longer than $8000 \angstrom$.  For our intrinsically red sources, this choice does not affect the spectra.  However, the telluric A0V and White Dwarf calibrators are very blue, causing strong second order overlap.  For the observations with this filter, we generated a synthetic spectral response correction.  We computed an average spectral response from the 2012 observing season, observed through the ``WB6300-9500'' filter, which has a short wavelength cutoff of 6300 $\angstrom$ and was therefore unaffected by second order overlap.  We divided out the published filter response and multiplied back by the Spectroscopic-2 filter curve.  We then applied this synthetic spectral response to the affected 2010 and 2011 spectra.

\subsection{IMACS Multi-slit exposure 2D spectral extraction}
\subsubsection{Multi-object spectroscopy data reduction}
We reduced the multi-object spectroscopy with the facility data reduction package \texttt{Cosmos}, version 2-16\footnote{See \url{http://code.obs.carnegiescience.edu/cosmos}} \citep{2011PASP..123..288D}, which includes an algorithm for optimal sky subtraction \citep{2003PASP..115..688K}.  \texttt{Cosmos} maps each spectrum from the multi-slit exposure, beginning with the coarse optics model of the instrument.  \texttt{Cosmos} iteratively refines the mapping solution to produce flat-fielded, sky-subtracted, rectified, 2D spectra for each slit.  We wrote custom IDL scripts\footnote{All of the code used for this paper is available on GitHub: \url{https://github.com/BrownDwarf/BAADE}} to automatically run through the \texttt{Cosmos} cookbook for all of the multi-slit exposures.  Mapping solutions should be accurate to one-tenth of a pixel \citep{2011PASP..123..288D}, we found mapping solutions were typically $<0.3$ pixels.

The main challenge to extract 1D spectra from 2D spectra was to automatically identify low signal to noise ratio traces or handle traces distorted from poor 2D mapping.  The spectral trace delivered by the \texttt{Cosmos} 2D spectral mapping is characterized by the pixel position of the center, and a rotation from the pixel $x-y$ basis.  The lowest signal to noise ratio traces were assigned the average center positions and rotations of the brightest traces, which worked fairly well at matching the visually identified traces for low signal to noise ratio spectra.

\subsubsection{Performance of spectral type standards compared to literature}
We observed several spectral type standards to demonstrate that we can reproduce the observed spectra of known late type sources.  We observed an L2 dwarf, 2MASS J11553952-3727350 \citep{2008AJ....136.1290R}; an M9 dwarf, 2MASS J15101685-0241078 \citep{2008AJ....136.1290R}; an L0pec, J00100009-2031122 \citep{2007AJ....133..439C}; and two young M7.75 members of Chamaeleon I, ChaH$\alpha$1 and ChaH$\alpha$7 \citep{2004ApJ...602..816L}.  We also observed the pair of low-mass wide binary brown dwarfs OPH1622-2405 a and b \citep{2006PhDT.........2A,2007ApJ...659.1629L}.  Figure \ref{fig_oph1622_compare} overplots our observation of OPH1622-2405a with the spectrum provided by K. Luhman from \citet{2007ApJ...659.1629L}, with comparable spectral resolution of about 18 $\angstrom$.  We overplotted our spectra with published spectra.  The spectra are virtually indistinguishable, except that the spectra from Luhman was not telluric corrected, whereas our spectra were telluric corrected.  The vertical gray bands indicate regions of known atmospheric absorption.  The other standard star spectra show comparable levels of agreement between their existing published spectra and our observations.  In a few cases the spectral slopes disagreed slightly, pointing to wavelength-dependent slit losses either in the published spectra or ours.  Based on the excellent accord in our spectra and published spectra, we are confident in all of the spectral response corrections and extraction strategies we applied.  

\begin{figure}[ht!]
\centering
\includegraphics[scale=0.9]{chIMACS/figures/oph1622-2405a_compare}
\caption[Comparison of IMACS spectroscopy to a previously published spectrum of the same source]{Direct comparison spectra of the M7.25 young brown dwarf OPH1622-2405a, observed with IMACS from this work (solid blue line), and provided by K. Luhman (dotted black line, \citet{2007ApJ...659.1629L}).  The regions of telluric atmospheric absorption are indicated with light gray vertical bands.  \label{fig_oph1622_compare}}
\end{figure}

Single pixel defects were also conspicuous in some of our comparison spectra.  These defects are mostly uncorrected cosmic rays.  Prominent in low signal-to-noise ratio spectra, the cosmic rays are about 10 to 20 times the typical pixel-to-pixel variance.  It is easy to identify these as cosmic rays during visual comparison.  However, automated spectral analysis like equivalent width measurements and continuum estimation can be affected by these large outliers.  We employ robust algorithms (\emph{e.g.} median instead of mean) when available, and manually flag pixels when needed.

Of the 530 slits on the seven slit masks (Table \ref{tbl_IMACSslitMasks}), 505 spectra passed through spectral reduction with \texttt{Cosmos}.  The other 25 were not reduced mostly because they fell on chip gaps or were irretrievable.

%------------------------------------------------------------------------------------------
\section{Spectral and photometric analysis}
%------------------------------------------------------------------------------------------

\subsection{Spectral classification with \emph{The Hammer}}
We visually inspected the IMACS $I-$band spectra for evidence of broad-band molecular absorption in the region $6500<\lambda(\angstrom)<9000 $.  The broad FeH, VO, and TiO bands are characteristic of late M spectral types, and are conspicuous even in low resolution ($R\sim100$) low signal-to-noise ratio ($S/N > 5$) spectra.  To accelerate the process of visually classifying hundreds of spectra, we employed Version 1.2.5 of \emph{The Hammer} spectral typing code \citep{2007AJ....134.2398C,2011AJ....141...97W}.  \emph{The Hammer} first computes a machine spectral type based on several spectral type sensitive indices.  Then \emph{The Hammer} overlays user-selectable comparison spectra, which can be visually matched to the overall spectral shape.  \emph{The Hammer} does not have a mechanism for tuning the extent of wavelength-dependent extinction, so since some of our sources are reddened, matches to the observed spectra may lead us to overestimate the spectral type.  This overestimate is okay, since we only use this first pass as a coarse filter for both signal-to-noise ratio ($S/N >5$), and overall spectral shape (consistent with M0 or later).  If the spectral shape is inconsistent with an M or L spectral type, we discard the spectrum from our follow up sample.  If there is evidence for an M or L spectral shape, we assign a coarse spectral type that we will later use to coarsely estimate the extinction.

\subsection{ Whittling down from 505 spectra to 71 late-type sources}
The 505 reduced spectra covered 416 unique targets.  The redundant observations came from the target reappearing on different masks in different years, since the slit mask fields occasionally overlapped.  In total, 337 were observed only once, 69 were observed twice, and 10 were observed three times.  None were observed more than three times.  We elected \emph{not} to coadd these spectra since many duplicate observations had one spectrum that was much higher signal to noise ratio than the other and repeats allowed us to look at variability (\emph{e.g} H$\alpha$) from year-to-year.  The duplicates allowed us to quantify repeatability and variance in our spectral estimators.  

We reduced the number of observed spectra from 505 to 92 by rejecting spectra that do not show evidence of an M or L spectral classification.  We matched the 92 M or L spectra with their 71 unique targets and picked the best spectrum.  These 71 sources define the follow up sample.  

More than half of the spectra that were rejected had $S/N < 8$.  About 5 of these rejects were due to the most egregious cases of instrumental artifacts in the MOS spectral mapping.  In those cases, we went back to visually inspect the 2D trace to convince ourselves that the traces were inconsistent with M or L spectral types.  Slitmask \texttt{ophwes} (Table \ref{tbl_IMACSslitMasks}) exhibited a large fraction of unusable spectra since since it only received $\sim45$ minutes of on-sky observing time due to a weather closure.  The lion's share of the rejects were spectra consistent with FGK stars (evinced by calcium triplet lines), or unknown interlopers.

\subsection{Extinction}
Reddening most likely leads us to conclude a spectral type 1 to 3 subtypes later than we ought to \citep{2010A&A...515A..75A}, assuming an $A_V\lesssim10$.  We assign an intrinsic $J-H$ for each of the 71 follow up sources based on their first pass coarse spectral type from \emph{The Hammer}.  We use $J-H$, since some sources were missing $I-$ band measurements.  We calculated the median and standard deviation of the intrinsic $J-H$ as a function of spectral subtype from M0$-$L5 using data from field dwarfs \citep{2011AJ....141...97W,2012ApJS..201...19D}.  We converted between 2MASS and MKO photometric systems where appropriate.  The spread in intrinsic $J-H$ is fairly large ($\sim0.1$), which propagates into our uncertainty in the reddening estimate.  Sources with $J-H$ bluer than field dwarfs for their spectral type were assigned zero extinction.  Otherwise, an $A_J$ was estimated from the $E(J-H)$.  $A_J$ was converted to $A_V$ assuming an $R_V=3.1$.  The range in $A_V$ is about $0 - 7$, which is consistent with the $A_V$ spread observed towards this region with 2MASS \citep{2008A&A...489..143L}.  The typical uncertainties were about $\sigma_{A_V}\sim1.0$ which includes the contribution from the uncertainty in the intrinsic $J-H$ and photometric uncertainties.

We revised our extinction estimate for the sources identified as members.  For these sources we de-redden to $J-H$ as a function of spectral type reported in \citet{2013ApJS..208....9P}.  We interpolated between points in cases of non-integer spectral subtypes.

\subsection{Second pass spectral type}
We dereddened all spectra by their estimated extinctions.  The dereddened spectra were once again visually classified with \emph{The Hammer}.  Figure \ref{fig_dereddened_comparison} shows the visual spectral type assigned before and after de-reddening.  The dashed line shows the 1-to-1 line obeyed by spectra of modest or no reddening.  A small 0.1 subtype jitter has been added to see the overlapping plot markers.  Sources with $A_V >3 $ initially show as high as 3 spectral subtype misclassification based on the de-reddened spectra.

\begin{figure}[ht!]
\centering
\includegraphics[scale=0.5]{chIMACS/figures/dereddened_spt_comparison}
\caption[Effect of reddening on spectral classification]{Comparison of first- and second- pass visual spectral classification of the 92 spectra with evidence of M or L spectral classification.  The one-to-one line is shown in long dashes.  Jitter has been added to the spectral subtypes to ease comparison of multiple overlapping sources.  The color bar indicates the amount by which each spectrum was de-reddened.  \label{fig_dereddened_comparison} }
\end{figure}

For most sources, the second pass spectral type is the final spectral type we assign.  Since the $J-H$ value to which we deredden is fairly constant from M4$-$M9, there is little to gain by re-estimating the extinction from the revised spectral type.

\subsection{Spectral comparison sample}
We assembled a spectral comparison sample to quantify the gravity-sensitivity of spectral diagnostics as a function of spectral type.  The spectra of dwarfs, giants, and young stars show differences because of their different surface gravities \citep{2012AJ....143..114S}.  A sufficiently large comparison sample will allow the quantification of the spread in spectral indices as a function of spectral type (proxy for temperature) and surface gravity (proxy for age).  The scatter in the spectral indices is important to assign uncertainties in our final age and membership classifications.

We constructed the spectral comparison sample from the RIZzo database \citep{cruz_kelle_2014_10721}, which includes 803 sources- 135 giants, 29 intermediate gravity sources \citep{2009AJ....137.3345C}, and 639 other sources which are mostly old field brown dwarfs or very low mass stars, but contains a few other interlopers and peculiar spectra.  Since the vast majority of the 639 other sources are field brown dwarfs, we treat all of them as such.  We crossmatched the RIZzo sample with Simbad to verify their spectral classifications.  A minuscule fraction of the sources had spectral types discordant by more than 1 subtype.  We made no attempt to remove these sources, since we are only using the aggregate properties of the comparison sample.

\subsection{Gravity sensitive indices}
The gravity sensitive Na~I line at 8200 \angstrom$\;$ is one indicator of whether our candidate young brown dwarfs are indeed young \citep{1999ApJ...525..466L,2007AJ....134.2398C,2009AJ....137.3345C}.  Figure \ref{fig_NaI_EW} shows the spectral type versus pseudo equivalent width of the Na~I line.  The continuum is ill-defined at low spectral resolution in our heavily line-blanketed spectra in the wavelength region of interest.  Nevertheless, the index clearly distinguishes between dwarfs and giants taken from the RIZzo comparison sample.  The dwarfs (red circles in Figure \ref{fig_NaI_EW}) demonstrate a pseudo equivalent width of about 6 $\angstrom$ by M8 spectral type, while giants (blue circles in Figure \ref{fig_NaI_EW}) reach values of about -2 $\angstrom$ at M8.  The low spectral resolution and choice of feature indices yield negative Na~I pseudo equivalent widths for the giants for spectral types later than M4.

\begin{figure}[ht!]
\centering
\includegraphics[scale=0.6]{chIMACS/figures/NaI_EW}
\caption[Na~I 8190 equivalent width as a function of spectral type from M0 to L5]{Na~I 8190 equivalent width as a function of spectral type from M0 to L5.  The sequence shows a clear bifurcation, splitting into groups of relatively high surface gravity (dwarfs, red points at the top), and low surface gravity (giants, blue points at the bottom).  Objects classified as intermediate gravity \citep{2009AJ....137.3345C} are shown as purple circles.  Young 10$-$20 Myr sources in the Beta Pictoris moving group (BPMG, \citet{2012AJ....143..114S}) are shown as large orange circles.  The dotted black line shows the median of the field dwarfs in Rizzo.  The solid black line excludes 90\% of the field objects, and defines our criterion for Na~I EW indicative of youth.  The green squares are young brown dwarf candidates from this work that meet the youth criterion.  The brown triangles are sources that do not meet the criterion for youth.\label{fig_NaI_EW} }
\end{figure}


Based on the observed Na~I index alone, thirty-one sources (green boxes)in our survey are consistent with low surface gravity suggestive of youth.  We adopted the continuum and feature indices defined by \citet{2012AJ....143..114S}, a central wavelength $\lambda_c= 8190 \angstrom$ and measurement window $\delta \lambda = 22 \angstrom$.  We assigned an uncertainty $\sigma_{EW}$ to the equivalent width by adding in quadrature the uncertainty in the mean of the feature and continuum values.  We defined a Na~I youth detection as 3$\sigma_{EW}$ below the bottom 10\% of Na~I equivalent width values of RIZzo dwarfs for a given spectral type.  The 10\% boundary is shown as a solid line in Figure \ref{fig_NaI_EW}.  We also overplot the Na~I EWs of Beta Pic moving group members tabulated in \citet{2012AJ....143..114S}.  Our Na~I EW threshold cuts out some sources in the 10-20 Myr Beta Pic moving group \citep{2012AJ....143..114S} at spectral types M3 and earlier.  Our spectra demonstrated a range of spectral types from M0 to L0.  The Na~I EW is ineffective at distinguishing gravity for M0 to M3 at our spectral resolution.  We also compared the Na~I EWs for our sample to those for members of Upper Scorpius \citep{2006AJ....131.3016S}.  We converted the ratio computed in their Table 1 to an equivalent width, and found a range of about 0.6$-$4.2 $\angstrom$ for spectral types M5 to M8, comparable to the range we see. 


\subsection{H$\alpha$ emission}
We computed the $H\alpha$ equivalent width using a 22 $\angstrom$ width centered on 6563 $\angstrom$, with continuum indices from 6530$-$6540 and 6570$-$6580 $\angstrom$.  We assigned an uncertainty to the equivalent width in the same way as we did for Na~I- adding the uncertainty in the means of feature and continua values in quadrature.  We report the measured H$\alpha$ equivalent width in Table \ref{tbl_members_phys}.  The presence of H$\alpha$ alone is not enough to establish a source as young, since $>40\;$\% of field M dwarfs later than M4 exhibit H$\alpha$ emission indicative of activity, with the fraction active approaching 80\% by spectral type M9 \citep{2011AJ....141...97W}.  


\subsection{Effective Temperatures}
We use the \cite{2003ApJ...593.1093L} effective temperature scale for spectral types M1-M9.  For M0 we assume $T_{\mathrm{eff}}= 3770$ \citep{2013ApJS..208....9P}.  We ignore sources with spectral types earlier than M0.  For sources with non-integer spectral substypes, we linearly interpolate between adjacent integer subtypes.  We assign an uncertainty in temperature of 150 K which is the uncertainty of the spectral classification ($\pm$1 subtype).  The systematic uncertainty in the temperature scale is even larger, as noted by \citet{2013ApJS..208....9P} and \citet{2010ApJS..186...63R}.

\subsection{Luminosities}
We computed luminosities for all 71 sources with late M or L spectral types.  We computed absolute magnitudes assuming a distance of 125 pc \citep{2008ApJ...675L..29L}, and using the apparent $J-$ magnitudes corrected for extinction and $J-$band bolometric corrections.  For spectral types M5 or earlier we used $BC_J$ reported by \citep{2013ApJS..208....9P}.  For spectral types M5.5 and later we used the mean $BC_J$ per spectral type bin reported in \citet{2002AJ....124.1170D}.  We linearly interpolated between data points for fractional spectral subtypes not reported in either \citet{2002AJ....124.1170D} or \citet{2013ApJS..208....9P}.


\subsection{Extinction and position in the HR Diagram}
\label{sec_HRD}
Equipped with derived effective temperatures and luminosities, we placed all 71 late type sources on an HR diagram.  Figure \ref{fig_HRD_NaI_Ex} shows the HR diagram with the pre-main sequence (PMS) evolutionary model tracks of \citep{1998A&A...337..403B,2002A&A...382..563B}.  The dotted lines are logarithmically spaced model isochrones from 1 Myr to 1 Gyr.  The solid black lines show the evolutionary track for sources down to 20 $M_{Jup}$.  The key idea from this Figure is that our sample is split into two populations- those below the main sequence and those above the main sequence.  Below the main sequence, sources are likely to be more distant background dwarfs.  Alternatively, a source can have a position in the HR diagram below the main sequence if it has a nearly edge-on disk.  These sources would have evidence for a mid-IR excess.  Sources OPH\_349 and OPH\_2 are potential edge-on disk sources.  Sources above the main sequence could also possibly be in the foreground, so membership assignment also requires evidence of low surface gravity from the optical spectroscopy.  We define 16 sources define as members, based on a holistic review of each object in terms of its position on the HR diagram, indication of extinction, evidence of strong H$\alpha$ emission, previous characterizations from the literature, presence of mid-IR excess emission indicative of a disk, and presence of weak Na~I EW.  The holistic approach means that all pieces of available evidence were considered together, and membership was decided taking into account missing or noisy data.  Tables \ref{tbl_members_phot} and \ref{tbl_members_phot_mid_IR} list the near-IR photometry and mid-IR photometry for the 16 sources that we define as members.  Table \ref{tbl_members_phys} lists their derived properties.

\begin{landscape}
	%%%%%%%%%%%%%%%%%%%%%%%%%%%%%%%%%%%%%%%%
% TABLE - Scores
%%%%%%%%%%%%%%%%%%%%%%%%%%%%%%%%%%%%%%%%
\begin{deluxetable}{lrrp{4cm}cccccc}
\tablecolumns{10}
\tabcolsep=0.11cm
\tabletypesize{\footnotesize}
\tablecaption{Photometry from this work\label{tbl_members_phot}}
\tablewidth{0pt}
\tablehead{
\colhead{Name} &
\colhead{RA} &
\colhead{DEC} &
\colhead{Alt. Names} &
\colhead{Ref.} &
\colhead{$Q$} &
\colhead{$I$} &
\colhead{$J$} &
\colhead{$H$} &
\colhead{$K$} } 
\startdata

 2mx\_10919 &  245.45202 & -23.67426 &  2M J16214848$-$2340273, [AKC2006] 9, [EDJ2009] 740 &  (1,2) & -0.91$\pm$0.15 &  17.50$\pm$0.01 &  13.59$\pm$0.03 &  12.34$\pm$0.03 &  11.69$\pm$0.02 \\
 2mx\_12644 &  245.56067 & -23.78650 &  \ldots &  \ldots & -0.27$\pm$0.07 &  16.19$\pm$0.01 &  12.85$\pm$0.03 &  11.89$\pm$0.02 &  11.41$\pm$0.03 \\
 2mx\_14856 &  245.68725 & -23.28707 &  2M J16224494-2317134, [AKC2006] 13, [EDJ2009] 748 &  (1,2,6) &  \ldots &  17.07$\pm$0.01 &  13.68$\pm$0.03 &  12.74$\pm$0.02 &  12.25$\pm$0.03 \\
 2mx\_10360 &  245.41609 & -23.19431 &  \ldots &  \ldots & -0.85$\pm$0.05 &  15.75$\pm$0.01 &  12.50$\pm$0.03 &  11.57$\pm$0.02 &  11.12$\pm$0.02 \\
   OPH\_673 &  245.42743 & -23.60256 &  \ldots &  \ldots & -0.91$\pm$0.07 &  19.68$\pm$0.01 &  15.04$\pm$0.01 &  13.97$\pm$0.01 &  13.25$\pm$0.01 \\
   OPH\_674 &  245.59602 & -24.11968 &  \ldots &  \ldots & -0.67$\pm$0.01 &  18.96$\pm$0.01 &  14.80$\pm$0.01 &  13.83$\pm$0.01 &  13.18$\pm$0.01 \\
   OPH\_678 &  245.31862 & -24.29600 &  \ldots &  \ldots & -0.69$\pm$0.01 &  18.42$\pm$0.01 &  14.81$\pm$0.01 &  13.96$\pm$0.01 &          \ldots \\
   OPH\_675 &  245.31889 & -23.60123 &  \ldots &  \ldots & -0.86$\pm$0.07 &  17.35$\pm$0.01 &  13.77$\pm$0.01 &  12.94$\pm$0.01 &          \ldots \\
  2mx\_8891 &  245.32603 & -23.18708 &  \ldots &  \ldots & -1.09$\pm$0.05 &  15.33$\pm$0.01 &  12.25$\pm$0.03 &  11.43$\pm$0.02 &  10.96$\pm$0.02 \\
 2mx\_14049 &  245.64313 & -23.38955 &  \ldots &  \ldots &         \ldots &  16.02$\pm$0.01 &  12.45$\pm$0.02 &  11.64$\pm$0.02 &  11.23$\pm$0.02 \\
   OPH\_679 &  245.25929 & -23.97774 &  2M J16210222$-$2358395 & (3) & -0.73$\pm$0.01 &  18.27$\pm$0.01 &  14.63$\pm$0.01 &  13.79$\pm$0.01 &  13.25$\pm$0.01 \\
   OPH\_681 &  245.81346 & -23.78497 &  \ldots &  \ldots & -1.09$\pm$0.06 &          \ldots &  15.10$\pm$0.01 &  14.21$\pm$0.01 &  13.56$\pm$0.01 \\
   OPH\_349 &  245.49823 & -23.26736 &  [JJK2008] J162159.55$-$231602.5 &  (4) & -0.58$\pm$0.02 &  19.25$\pm$0.01 &  16.32$\pm$0.01 &  15.67$\pm$0.01 &  15.36$\pm$0.01 \\
 OPH\_12081 &  245.39963 & -23.91767 &  2M J16213591$-$2355035, [SCH2006] J16213591-23550341 &  (5) & -1.06$\pm$0.01 &  17.10$\pm$0.01 &  13.79$\pm$0.01 &  13.16$\pm$0.01 &  12.72$\pm$0.01 \\
     OPH\_2 &  245.56548 & -23.50553 &  \ldots &  \ldots & -1.01$\pm$0.12 &  20.66$\pm$0.07 &  17.63$\pm$0.02 &  17.09$\pm$0.02 &  16.90$\pm$0.04 \\
   OPH\_672 &  245.56064 & -23.95042 &  \ldots &  \ldots & -1.03$\pm$0.03 &  18.48$\pm$0.01 &  14.62$\pm$0.01 &  13.91$\pm$0.01 &  13.36$\pm$0.01 \\
\enddata

\tablecomments{$Q$ is the degree of water absorption in the $W-$filter band. The source names containing ``2mx'' are from the 2MASS $JH$ catalog.  The sources with ``OPH'' are from the ISPI $JH$ catalog.}
\tablerefs{
(1)~\citet{allers06}
(2)~\citet{2009ApJS..181..321E}
(3)~\citet{2012ApJ...758...31L}
(4)~\citet{2008ApJ...683..822J}
(5)~\citet{2006AJ....131.3016S}
(6)~\citet{2012ApJ...755...67H}
}

\end{deluxetable}
\end{landscape}

\begin{landscape}
	%%%%%%%%%%%%%%%%%%%%%%%%%%%%%%%%%%%%%%%%
% TABLE - Scores
%%%%%%%%%%%%%%%%%%%%%%%%%%%%%%%%%%%%%%%%
\begin{deluxetable}{lccccc}
\tablecolumns{6}
\tabcolsep=0.11cm
\tabletypesize{\footnotesize}
\tablecaption{Mid-IR Photometry from this work\label{tbl_members_phot_mid_IR}}
\tablewidth{0pt}
\tablehead{
\colhead{Name} &
\colhead{$[3.6]$} &
\colhead{$[4.5]$} &
\colhead{$[5.8]$} &
\colhead{$[8.0]$} &
\colhead{$[24]$} 
}
\startdata
 2mx\_10919 &  10.89$\pm$0.05 &  10.44$\pm$0.05 &  10.02$\pm$0.01 &   9.13$\pm$0.01 &   4.88$\pm$0.10 \\
 2mx\_12644 &  10.95$\pm$0.05 &  10.84$\pm$0.05 &  10.71$\pm$0.01 &  10.66$\pm$0.01 &  10.35$\pm$0.83 \\
 2mx\_14856 &  11.58$\pm$0.05 &  11.26$\pm$0.05 &  11.02$\pm$0.01 &  10.35$\pm$0.01 &   7.83$\pm$0.11 \\
 2mx\_10360 &  10.74$\pm$0.05 &  10.64$\pm$0.05 &  10.55$\pm$0.01 &  10.47$\pm$0.05 &  10.07$\pm$0.44 \\
   OPH\_673 &  12.53$\pm$0.05 &  12.39$\pm$0.05 &  12.18$\pm$0.01 &  12.19$\pm$0.01 &          \ldots \\
   OPH\_674 &  12.54$\pm$0.06 &  12.40$\pm$0.05 &  12.26$\pm$0.01 &  12.26$\pm$0.01 &          \ldots \\
   OPH\_678 &  12.81$\pm$0.05 &  12.74$\pm$0.06 &  12.57$\pm$0.07 &  12.45$\pm$0.01 &            $<$8.5 \\
   OPH\_675 &  11.87$\pm$0.05 &  11.71$\pm$0.05 &  11.61$\pm$0.01 &  11.59$\pm$0.01 &   9.94$\pm$0.49 \\
  2mx\_8891 &  10.54$\pm$0.05 &  10.44$\pm$0.05 &  10.32$\pm$0.05 &  10.28$\pm$0.05 &  10.49$\pm$0.88 \\
 2mx\_14049 &  10.80$\pm$0.06 &  10.67$\pm$0.05 &  10.63$\pm$0.02 &  10.53$\pm$0.01 &  10.10$\pm$0.59 \\
   OPH\_679 &  12.65$\pm$0.05 &  12.52$\pm$0.05 &  12.37$\pm$0.01 &  12.36$\pm$0.01 &            $<$9.6 \\
   OPH\_681 &  12.95$\pm$0.05 &  12.73$\pm$0.05 &  12.52$\pm$0.02 &  12.47$\pm$0.02 &  10.55$\pm$0.91 \\
   OPH\_349 &  14.96$\pm$0.06 &  13.85$\pm$0.06 &  12.53$\pm$0.01 &  11.96$\pm$0.01 &   7.17$\pm$0.10 \\
 OPH\_12081 &  12.20$\pm$0.05 &  12.12$\pm$0.05 &  11.94$\pm$0.01 &  11.89$\pm$0.01 &  10.25$\pm$0.75 \\
     OPH\_2 &  16.05$\pm$0.08 &  15.07$\pm$0.08 &  13.85$\pm$0.02 &  12.25$\pm$0.02 &   8.88$\pm$0.20 \\
   OPH\_672 &  12.72$\pm$0.05 &  12.65$\pm$0.06 &  12.51$\pm$0.01 &  12.44$\pm$0.01 &            $<$8.1 \\
\enddata
%\tablecomments{}
%\tablerefs{}
\end{deluxetable}
\end{landscape}

\begin{figure}[ht!]
\centering
\includegraphics[scale=0.6]{chIMACS/figures/HRD_zoom.pdf}
  \caption[Pre main sequence HR diagram with young very low mass stars and brown dwarfs]{Pre main sequence HR diagram indicating membership of young very low mass stars and brown dwarfs to \emph{Ophiuchus}.  The green circles are members based in part on their position in this diagram.  The red squares are sources without evidence for membership, or evidence of non-membership.  Sources OPH\_349 and OPH\_2 are edge-on disks, based on the presence of mid-IR excess, suppressed near-IR and optical colors, and their apparent position below the main sequence in this diagram. \label{fig_HRD_NaI_Ex} }
\end{figure}

\subsection{Spectral Energy Distributions}
\label{sec_SED}

We explored the diversity of mid-IR excess emission properties by constructing spectral energy distributions (SEDs).  Figures \ref{fig_SEDs_9panel} and \ref{fig_SEDs_7panel} show the SEDs for the 16 members in our sample.  Many of the sources identified as young, low-mass objects do not appear to have mid-IR excess, despite meeting the disk criterion for selection.  Often, such sources met the selection criterion because they were detected at 24 $\mu$m.  When designing the survey, we targeted the lowest luminosity young brown dwarfs, which should not have detectable photospheres at 24 $\mu$m at the distance of Ophiuchus.  The sources we detected are typically warmer and more luminous than our prediction for M8 sources, so we can detect photospheric 24 $\mu$m emission in some of them.

\begin{figure}[ht!]
\centering
\includegraphics[scale=0.5]{chIMACS/figures/SEDS_9panel}
\caption[Spectral energy distributions of nine members of \emph{Ophiuchus}]{Spectral energy distributions of 9 members with spectral types M5$-$M6.  The gray circles show the observed photometry.  The green squares show the photometry corrected for the extinction value derived from the first-pass spectral typing.  Both the observed and extinction-corrected photometric data have been normalized by the extinction-corrected $H-$band measurement.  The red dotted line shows the blackbody with temperature equal to $T_{\mathrm{eff}}$ for each source.  The observed photometry is sometimes below the blackbody photosphere, since the photospheric spectra for the cool sources differ substantially from blackbody sue to molecular species in their atmospheres.  \label{fig_SEDs_9panel} }
\end{figure}

\begin{figure}[ht!]
\centering
\includegraphics[scale=0.5]{chIMACS/figures/SEDS_7panel}
\caption[Spectral energy distributions of seven members of \emph{Ophiuchus}]{Spectral energy distributions of 7 members with spectral types M6$-$M8.5.  The plots have the same symbols and scaling as in Figure \ref{fig_SEDs_9panel}.  \label{fig_SEDs_7panel} }
\end{figure}

%------------------------------------------------------------------------------------------
\section{Results}
%------------------------------------------------------------------------------------------
\subsection{Whittling down from 71 to 16}
We adopted a holistic approach to assigning membership to the 71 targets with late-type spectra to arrive at 16 sources with confident evidence for membership.  An additional 3 sources are consistent with membership, but have spectral types M4 or earlier, which makes their Na~I equivalent widths difficult to distinguish between dwarfs and YSOs.  The remainder of targets have evidence for high gravity features and/or are below the main sequence line and are therefore deemed dwarf contaminants.  Table \ref{tbl_members_phys} lists the derived properties for the final sample of 16 members.  

\begin{landscape}
	%%%%%%%%%%%%%%%%%%%%%%%%%%%%%%%%%%%%%%%%
% TABLE - Scores
%%%%%%%%%%%%%%%%%%%%%%%%%%%%%%%%%%%%%%%%
\begin{deluxetable}{lccccccccc}
\tablecolumns{10}
\tabcolsep=0.11cm
\tabletypesize{\footnotesize}
\tablecaption{Spectral classification and derived properties of our sample\label{tbl_members_phys}}
\tablewidth{0pt}
\tablehead{
\colhead{Name} &
\colhead{Priority} &
\colhead{Sp. Type} &
\colhead{$A_V$} &
\colhead{$T_{\mathrm{eff}}$} &
\colhead{$\log L/L_{\odot}$} &
\colhead{H$\alpha$ EW} &
\colhead{Na~I EW} &
\colhead{SED Class} &
\colhead{Night} \\
\colhead{} &
\colhead{} &
\colhead{} &
\colhead{} &
\colhead{K} &
\colhead{} &
\colhead{$\angstrom$} &
\colhead{$\angstrom$} &
\colhead{} &
\colhead{YYYYMMDD} 
}
\startdata
 2mx\_10919 &                   1 &       M5 &  7.1 &  3130 &    -1.4 &     -126.8 &  22.0$\pm$0.3 &        II &    20120513 \\
 2mx\_12644 &                   8 &       M5 &  4.1 &  3130 &    -1.4 &       -7.9 &   2.1$\pm$0.1 &       III &    20120513 \\
 2mx\_14856 &                   3 &    M5.75 &  4.0 &  3020 &    -1.8 &       -8.3 &   2.4$\pm$0.1 &        II &  20120513-4 \\
 2mx\_10360 &                   7 &       M5 &  3.8 &  3130 &    -1.3 &       -6.5 &   2.8$\pm$0.1 &       III &  20120513-4 \\
   OPH\_673 &                   5 &     M8.5 &  4.1 &  2560 &    -2.3 &      -13.9 &   4.3$\pm$0.1 &       III &    20120513 \\
   OPH\_674 &                   5 &       M7 &  4.4 &  2880 &    -2.2 &      -13.2 &   3.5$\pm$0.1 &       III &    20100617 \\
   OPH\_678 &                   5 &    M5.75 &  3.6 &  3020 &    -2.3 &       -4.8 &   3.5$\pm$0.1 &       III &    20120514 \\
   OPH\_675 &                   5 &     M5.5 &  3.4 &  3060 &    -1.9 &       -7.2 &   2.8$\pm$0.1 &       III &    20120514 \\
  2mx\_8891 &                   7 &     M5.5 &  2.9 &  3060 &    -1.3 &       -6.1 &   2.2$\pm$0.1 &       III &  20120513-4 \\
 2mx\_14049 &                   8 &     M5.5 &  2.8 &  3060 &    -1.4 &       -6.7 &   2.3$\pm$0.1 &       III &  20120513-4 \\
   OPH\_679 &                   5 &     M6.5 &  3.2 &  2940 &    -2.2 &       -7.4 &   3.6$\pm$0.2 &       III &    20120513 \\
   OPH\_681 &                   5 &     M8.5 &  2.3 &  2560 &    -2.5 &       -6.5 &   1.0$\pm$0.2 &       III &    20110707 \\
   OPH\_349 &                   4 &       M6 &  1.5 &  2990 &    -3.1 &       -8.6 &   2.1$\pm$0.1 &        EO &    20100617 \\
 OPH\_12081 &                   6 &       M7 &  1.0 &  2880 &    -2.1 &       -7.9 &   2.8$\pm$0.1 &       III &    20110707 \\
     OPH\_2 &                   1 &       M6 &  0.4 &  2990 &    -3.7 &      -44.5 &   1.7$\pm$0.1 &        EO &    20120514 \\
   OPH\_672 &                   5 &    M8.25 &  0.9 &  2630 &    -2.5 &      -16.6 &   1.9$\pm$0.2 &       III &    20110707 \\
\enddata

\tablecomments{Typical uncertainties are about 0.2 in $A_V$, 150 K in $T_{\mathrm{eff}}$, 0.2 dex in luminosity, and 0.1 $\angstrom$ in EW.  The SED Class refers to the mid-IR slope, with ``EO'' indicating edge-on disks.}
%\tablerefs{}

\end{deluxetable}
\end{landscape}

\subsection{Discussion of individual sources}
\label{sec_individual_sources}

\begin{figure}[ht!]
\centering
\includegraphics[scale=0.6]{chIMACS/figures/IMACS_spectra_M0_M6}
\caption[IMACS spectra of 10 very low mass YSOs or young brown dwarfs towards \emph{Ophiuchus} with spectral types M5-M6]{IMACS spectra of 10 very low mass YSOs or young brown dwarfs towards \emph{Ophiuchus} with spectral types M5-M6.  The spectra have been Gaussian smoothed with a 7$\angstrom$ kernel bandwidth.  All spectra were corrected for telluric absorption.  The spectra are normalized at 7500 $\angstrom$ and offset by 1 in the vertical direction. \label{fig_IMACS_spectra1} }
\end{figure}


\begin{figure}[ht!]
\centering
\includegraphics[scale=0.65]{chIMACS/figures/IMACS_spectra_M6_M8p5}
\caption[IMACS spectra of 5 young brown dwarfs with spectral types M6$-$M8.5]{IMACS spectra of 5 young brown dwarfs with spectral types M6$-$M8.5.  Same scaling as in Figure \ref{fig_IMACS_spectra1}. \label{fig_IMACS_spectra2} }
\end{figure}

Figures \ref{fig_IMACS_spectra1} and \ref{fig_IMACS_spectra2} show the spectra of the M5$-$M6 and M6$-$M8.5 sources consistent with membership to \emph{Ophiuchus}.

2mx\_12644 shows a spectral shape comparable to the 2 Myr M5 source T50 \citep{2004ApJ...602..816L}, with comparably weak Na~I absorption.  It has modest H$\alpha$, although not substantially more than active M dwarfs \citep{2011AJ....141...97W}.  

2mx\_10919 was discovered in \citet{allers06} (their source 9), and was given a spectral type of M5.5 by \citet{2011ASPC..448..633G}.  Its IMACS spectrum is unluckily missing the portion of spectrum surrounding the 8200 $\angstrom$ Na~I line, but its youth and membership are firm based on the presence of mid-IR excess and significant H$\alpha$ emission, as well as the NIR spectrum from \citet{2011ASPC..448..633G}.  We issue 2mx\_10919 a spectral type of M5, based on its similarities with T50.  2mx\_10919 also has the largest estimated extinction of any source, supporting our interpretation that it is a member of \emph{Ophiuchus}.

2mx\_10360 has spectral features very similar to T50, so we assign an M5 spectral type, and membership to \emph{Ophiuchus} based on its weak Na~I absorption.

2mx\_8891 has a spectrum most similar to chxr84, an M5.5 member of Chamaeleon I \citep{2004ApJ...602..816L}, with comparable Na~I.  Its Na~I is much weaker than the M5.5 dwarf 2MASS J11245327+1322533 \citep{2003AJ....126.2421C}.

OPH\_675 has an M5.5 spectral type based on comparison to the same comparison sources as 2mx\_8891.  OPH\_675 has a slightly deeper appearance of the Na~I line than chxr84 does, but still much less than 2MASS J11245327+1322533, so we assign it youth and therefore membership.  Uncorrected telluric absorption in the comparison spectra could contribute to the slightly deeper appearance of Na~I in our spectra versus those of Luhman and RIZzo.  

2mx\_14049 looks almost identical to OPH\_675.

OPH\_678 looks most similar to ChaH$\alpha$6 \citep{2004ApJ...602..816L}, an M5.75.  OPH\_678 has slightly deeper Na~I than ChaH$\alpha$6 does, but much shallower than field M5.5 2MASS J11245327+1322533.  

2mx\_14856 was discovered by \citet{allers06} (their source 13), with spectral type M6 from \citet{2011ASPC..448..633G}.  It has mid-IR excess indicative of a disk.  Further, its weak Na~I and moderate H$\alpha$ support its youth.  Its IMACS spectral type is most similar to ChaH$\alpha$6, so we assign it spectral type M5.75.

OPH\_679 has a low signal-to-noise ratio spectrum.  Its spectrum is comparable to the Cha I M6.5, iso138 \citep{2004ApJ...602..816L}, with much less Na~I absorption than 2MASS J14450627+4409393 \citep{2003AJ....126.2421C}.  OPH\_679 is listed as an M5.25 Upper Sco member, source 2MASS J16210222$-$2358395 in \citet{2012ApJ...758...31L}, though no spectrum for it has been published.  

OPH\_673 also has a low signal-to-noise ratio spectrum.  Its spectral shape and Na~I absorption are comparable to the M8.5 KPNO06 \citep{1998AJ....115.2074B,2003ApJ...593.1093L}.  At this late spectral type, the dwarf comparison spectra exhibit Na~I comparable to their young counterparts, so assigning youth is more challenging.  Still, the Na~I line strength in OPH\_673 and KPNO06 is noticeably shallower than M8.5 dwarf 2MASS J16141557+8211327 \citep{2007AJ....133..439C}.  We tentatively assign it as a member.

OPH\_674 resembles Cha I M7 11123099-7653342 \citep{2004ApJ...602..816L}.  The Na~I line is suficiently weaker than 2MASS J04365019-1803262 \citep{2007AJ....133..439C}, so we assign OPH\_674 as amember.  

The spectrum of OPH\_681 is most similar to M8.5 KPNO06 \citep{1998AJ....115.2074B,2003ApJ...593.1093L}, and firmly weaker than M8.5 V 2MASS J13235206+3014340 \citep{2007AJ....133..439C}.  

OPH\_1 and OPH\_3 were not above the main sequence for their derived $T_{\mathrm{eff}}$ and $L_{\mathrm{bol}}$, but exhibited apparent mid-IR excess in IRAC band 4.  We examined the IRAC band 4 imaging and found that the IRAC4 detections were consistent with spurious structure from cloud nebulosity.  OPH\_3 exhibits strong Na~I and K~I absorption for its M3 spectral type.  The Na~I absorption cannot be reliably distinguished between OPH\_1 and dwarfs of the same spectral type.  Nevertheless, their underluminosity and refuted evidence for a disk lead us to mark these sources as background objects (given their extinction).

OPH\_2 has dramatically rising mid-IR excess emission.  The low signal-to-noise ratio spectrum of OPH\_2 is most similar to that of Cha I M6 chsm1982 \citep{2004ApJ...602..816L}, and has a clear indication of youth from its weak Na~I line.  The moderate H$\alpha$ supports its youth.  Its apparent position below the main sequence leads us to demarcate OPH\_2 as an edge-on disk, and therefore a member.

By the same logic applied to OPH\_2, OPH\_349 is also an edge-on disk.  Specifically, its prominent mid-IR excess and apparent position below the main sequence are strong evidence for being edge-on disks.  The spectrum is consistent with an M6 spectral type classification, with Na~I comparable to that of CHSM1982.  The H$\alpha$ is weak in this source.

OPH\_12081 was identified as a M6 member of Upper Scorpius by \citet{2006AJ....131.3016S}.  We assign it a spectral type of M7 based on its similarity to Cha I M7 11123099-7653342 \citep{2004ApJ...602..816L}.

OPH\_672 is most similar to M8.25 KPNO 7 \citep{1998AJ....115.2074B,2003ApJ...593.1093L}, with a similar amount of Na~I absorption.

Sources 2mx\_10500, 2mx\_10966, and 2mx\_9647 have spectral types M4 or earlier.  We leave these sources out of our member list since they are more likely low mass stars than brown dwarfs.

2mx\_9647 is the earliest spectral type source that meets either of our youth criteria.  It is consistent with a spectral type earlier than M1.  We assign it an M0 spectral type, though it could be a late K type.  There is no evidence of accretion, and the source does not have evidence for mid-IR excess.  The source is above the main sequence and demonstrates an $A_V$ of 3.8, so we reject the interpretation that it is a foreground object.  Still, we cannot confidently assign evidence of youth based on the spectrum, so we mark this source as a candidate.

2mx\_10966 has an M4 spectral type with modest H$\alpha$ emission.  The Na~I equivalent width is weak- more similar to Hn17 \citep{2004ApJ...602..816L}, than 2M J05102396-2800539 \citep{2007AJ....133..439C}, but only marginally so.  We mark this source as a candidate.

The youth of 2mx\_10500 is firm based on mid-IR excess, its weak Na~I equivalent width, and its $A_V\sim2.7$ and position above the HR diagram.  Its moderate H$\alpha$ lends support to its youth and therefore membership.  Its Na~I is weaker than M4V 2MASS J02193311+1416327 \citep{2003AJ....126.2421C}, and comparable to the 2 Myr M4 Hn17 \citep{2004ApJ...602..816L}.



\subsection{Edge-on disks}
Figures \ref{fig_SEDs_9panel} and \ref{fig_SEDs_7panel} include the conspicuously rising SEDs of sources OPH\_349 and OPH\_2.  These sources demonstrate mid-IR flux increasing with wavelength from 3.6 to 24 $\mu$m.  The sources fall below the main sequence in the HR diagram, indicating that these sources are most likely heavily-extincted edge-on disks with near-IR colors determined by a mixture of scattering and extinction.  The IMACS spectrum of OPH\_2 shows strong H$\alpha$ emission.  OPH\_349 is separated by 4" from OPH26571.  Figure \ref{fig_OPH_349_pair} shows the postage stamp images of the pair in different bands.  \cite{2008ApJ...683..822J} identified OPH\_349 during their search for deeply embedded YSOs.  This source is included in their list of candidate edge-on disks, since it exhibited very red mid-IR colors but no associated SCUBA flux.


\begin{figure}[ht!]
\begin{minipage}[b]{0.3\linewidth}
\centering
\includegraphics[scale=0.15]{chIMACS/figures/OPH_349_I_image}
\end{minipage}
\hspace{0.5cm}
\begin{minipage}[b]{0.3\linewidth}
\centering
\includegraphics[scale=0.15]{chIMACS/figures/OPH_349_IRAC1_image}
\end{minipage}
\hspace{0.5cm}
\begin{minipage}[b]{0.3\linewidth}
\centering
\includegraphics[scale=0.15]{chIMACS/figures/OPH_349_IRAC4_image}
\end{minipage}
  \caption[Images of an edge-on disk source with nearby source]{Images of OPH\_349 with the nearby OPH\_26571.  Images are from left to right: MOSAIC $I-$band, $[3.6]$, $[8.0]$. \label{fig_OPH_349_pair}}
\end{figure}

\subsection{Mid-IR colors of diskless members}

We looked at the distributions of the mid-IR slope of our sample to distinguish between the diskless and disk-bearing members.  We include in our analysis the mid-IR data of 187 Class I, II, and III Taurus members with spectral types later than M0 \citep{2010ApJS..186..111L}.  The large number of Taurus members allows us to construct a kernel density estimate in a plot of $[3.6]-[8.0]$ versus spectral type, shown in Figure \ref{fig_I1_I4_density}.  The marginal distribution of $[3.6]-[8.0]$ reveals two populations: a presumably diskless group of sources with $[3.6]-[8.0] \sim 0$ and disk-bearing sources with $[3.6]-[8.0] \sim 1.5$.  The valley in-between the disk-bearing and diskless sources can be understood as evidence for a short-lived phase of disk-clearing \citep{2009ApJS..181..321E}.  All of the \citet{allers06} sources (black circles in Figure \ref{fig_I1_I4_density}) were selected for evidence of mid-IR excess, they appear in the upper branch with $[3.6]-[8.0] \sim 1.5$.  Most of our sources (white squares in Figure \ref{fig_I1_I4_density}) appear in the lower branch with $[3.6]-[8.0] \sim 0$.  The source 2MASS J16214199$-$2313432, an M3 \citep{2011ASPC..448..633G}, has $[3.6]-[8.0]=0.79\pm0.2$ placing it in the valley between the two clusters of disk bearing and diskless objects.  This source might be the only ``transition disk'' out of the \citet{allers06} sample and our sample.  Despite being sensitive to them, we find no sources in this valley, a feature that extends into the sub-stellar regime.  We take this observation as evidence that whatever the mechanism for disk-clearing is, it proceeds similarly in the sub-stellar regime as it does for stellar objects. 

\begin{figure}[ht!]
\centering
\includegraphics[scale=0.6]{chIMACS/figures/I1_I4_density.pdf}
\caption[Mid-infrared slope as a function of spectral type]{Distribution of $[3.6]-[8.0]$ as a function of spectral type for young stellar and substellar objects in this work (white squares with black edges).  The distribution of 187 Class I, II, and III Taurus members with spectral types later than M0 \citep{2010ApJS..186..111L} is shown as the kernel density estimate contours in the background.  The black circles with white edges are the 8 sources towards \emph{Ophiuchus} from \citet{allers06} with spectral types from \citet{2011ASPC..448..633G}.  The white dashed line is constructed from Class III young stellar and substellar objects in $\eta$ Cha, $\epsilon$ Cha, and TWA \citep{2010ApJS..186..111L}, and represents the photospheric colors of young objects.  The two edge-on disks OPH\_2 and OPH\_349 are outliers in this plot.  The two disk-bearing sources are 2mx\_10919 and 2mx\_14856.  The other 12 members are diskless sources.  \label{fig_I1_I4_density} }
\end{figure}

We used the class III analogs in our sample to construct mid-IR photospheric colors of late-type sources.  \citet{2010ApJS..186..111L} construct the photospheric colors of young objects as a function of spectral type with evolved class III objects in $\eta$ Cha, $\epsilon$ Cha, and TWA\footnote{The original table with these values contains an error, see the erratum in \cite{2010ApJS..189..353L}}.  We extend the relationship of \citet{2010ApJS..186..111L} with more late-type sources by combining our sample with theirs.  The \citet{2010ApJS..189..353L} sample includes 7 Class III sources with spectral type later than M8: J04272799+2612052, J04274538+2357243, J04302365+2359129, J04354526+2737130, J04355143+2249119, J04373705+2331080, J04574903+3015195.  This work adds 3 new Class III sources with spectral types later than M8.  The colors are consistent with the \citet{2010ApJS..189..353L} sample, with a mean $[3.6]-[8.0]=0.3$ for M8.5 spectral types.  For comparison, M0$-$M2 sources have $[3.6]-[8.0]<0.1$.    Figure \ref{fig_midIR_results} shows a close-up on the mid-IR colors of old field dwarfs \citep{2006ApJ...651..502P}, young Taurus members \citep{2010ApJS..186..111L}, and \emph{Ophiuchus} members from \citep{allers06} and this work.  The increasing bifurcation between the background density of Taurus Class II and Class III sources as a function of wavelength can be understood as the mid-IR excess detaching from the photosphere.


\begin{figure}[ht!]
\centering
\includegraphics[scale=0.6]{chIMACS/figures/midIR_phot_results.pdf}
\caption[Mid-IR colors as a function of spectral type for our sample]{Mid-IR colors as a function of spectral type for our sample.  Young stellar and substellar objects in this work are white squares with black edges and error bars.  The distribution of 187 Class I, II, and III Taurus members with spectral types later than M0 \citep{2010ApJS..186..111L} is shown as the kernel density estimate contours in the background.  The black circles with white edges are the 8 sources towards \emph{Ophiuchus} from \citet{allers06} with spectral types from \citet{2011ASPC..448..633G}.  The white dashed line is constructed from Class III young stellar and substellar objects in $\eta$ Cha, $\epsilon$ Cha, and TWA \citep{2010ApJS..186..111L}, and represents the photospheric colors of young objects.  The photospheric colors of field dwarfs from \citet{2006ApJ...651..502P} are shown as blue diamonds.  The increasing dichotomy between the background density of Taurus members as a function of wavelength can be understood as the mid-IR excess detaching from the photosphere.  The Taurus members and \citet{allers06} members have not been corrected for extinction.  \label{fig_midIR_results} }
\end{figure}

%------------------------------------------------------------------------------------------
\section{Analysis}
%------------------------------------------------------------------------------------------
\subsection{Have we found diskless YBDs?}
One of the main results of this work is that we have found a large population of diskless young sources.  Taken at face value, these discoveries could indicate that 1) disk fractions are probably lower for low-mass stars/brown dwarfs than was previously thought, and 2) brown dwarfs are slightly more common than was previously thought.  We join our sample with the \citet{allers06} \emph{Ophiuchus} sample to construct a disk fraction for sources with spectral type later than M4.  There are four sources from \citet{allers06} with spectral type later than M4- their sources 9, 11, 13, and 14.  Source 11 (OPH1622$-$405) was identified as an Upper Sco member \citep{2007ApJ...659.1629L}, so we leave it out of consideration.  Sources 9 and 13 are common to our sample, and are our only disk-bearing sources, other than the edge-on disks.  So the total number of disk bearing objects towards this region of \emph{Ophiuchus} is 5.  The rest of our sources, excluding the edge-on disks, are diskless.  There are 12 diskless sources in our sample.  Two of those were previously identified as Upper Sco members \cite{2006AJ....131.3016S,2012ApJ...758...31L}, so we remove them from our comparison.  So we have 10 diskless sources towards this region of \emph{Ophiuchus}.  That gives us a disk fraction of 5/15, or 33\%.  

The average disk fraction for spectral types later than M4 is about 40-60\% in Taurus (1 Myr), Chamaeleon I (2-3 Myr), and IC348 (2-3 Myr) \citep{2012ARA&A..50...65L}.  Upper Scorpius ($\sim 11$ Myr) has a disk fraction of about $20-30$\% for spectral types later than M4.  Our derived disk fraction 33\% would be unusually low for \emph{Ophiuchus}, if we assume all of the objects are members of \emph{Ophiuchus}.  It is conceivable that some of our objects are members of the older population of Upper Sco, which would explain the absence of disks around such a large fraction of our sources.  However, not all of our diskless sources are likely to be members of Upper Sco, since the surface number density is too high for Upper Sco, and the Na~I equivalent widths are comparable to young Taurus members.  Allowing as many as three addtional sources to belong to Upper Sco would revise our disk fraction from 5/15 to 5/12, yielding a disk fraction of $\sim40$\%, which is comparable to the $\sim1-3$ Myr Taurus, Chamaeleon, and IC348.  

\subsection{What would we have selected if we had used selection criteria similar to other authors?}
The various surveys for young brown dwarfs use different combinations photometric systems, and different permutations of successive color cuts.  It is difficult to reproduce the selection of other surveys with our available data, unless we restrict ourselves to large survey data (like WISE, 2MASS, c2d).  Most of the sources in the spectroscopic sample that are not confirmed as members in our work are background sources, based on their apparent position below the main sequence.  This makes sense, since the $W-$ filter does not select based on youth, but only late spectral type.  

The main difference between our result and others' is that we find a large fraction of diskless sources.  Any survey that relies on mid-IR excess as a criterion for selection would miss a large fraction of our detections.

\subsection{Performance of the $W-$ filter}

We use our derived spectral types to evaluate to what extent the $W-$ filter $Q$ value correlates, as intended, with spectral type.  Figure \ref{fig_W_results} shows $Q$ as a function of spectral type for both our confirmed young members (green squares) and late-type non-members.  We fit a straight line to the members, which have higher precision on their spectral type classification than the non-members.  The fit shows a trend in the right direction: later type sources have larger $Q$ values on average.  We do not use the uncertainty in the $Q$ value in the straight-line fits.  The trend is even stronger in the same direction, if the measurement uncertainty is included in the fit.  Still, the large number of non-members that disobey the fit point to the role of measurement uncertainty both in $Q$ and in the coarse spectral typing of non-members with \emph{The Hammer}.  We find no late-type sources with $Q>0$ (excluding a known wide binary brown dwarf 1622-2405, which has been omitted from the figure, since its $W-$band photometry is likely affected by its companion.)  Most of the non-members and all but 1 of the members have $Q<-0.5$, which defined the lower boundary of our $W-$band selection method.  

\begin{figure}[ht!]
\centering
\includegraphics[scale=0.6]{chIMACS/figures/W_filter_results}
\caption[Performance of the $W-$filter $Q$ value as a function of spectral type]{Performance of the $W-$filter $Q$ value as a function of spectral classification.  The blue squares are members, the green squares are late-type non-members.  The left portion of the plot shows the probability density function for the parent catalog of 31348 sources with a $W-$filter measurement (red stepped histogram with fine binning centered at $Q=0.0$).  The probability density distribution of all the late-type non-members is shown as the blue stepped histogram with coarse binning. We have excluded the outlier 1622-2405, a known binary brown dwarf, from the figure.  The spectral type positions have a small ($\sim0.1$ spectral subtype jitter added to them to displace the points for legibility).\label{fig_W_results} }
\end{figure}


\subsection{What is the range of model-derived masses of all of the sources?}
Based on the position of the sources in the HR diagram in Figure \ref{fig_HRD_NaI_Ex} we can infer a mass and age for our sources.  From inspection of the HR diagram we see that the objects range in inferred ages from 1$-$80 Myr, and inferred masses from 20-200 Jupiter masses.  There are many perils associated with deriving ages and masses in this way.  This strategy assumes the models are accurate, and that secondary effects like accretion history have not occured.  We also assume that chromospheric activity is negligible \citep{2014ApJ...796..119S}.  The uncertainties in spectral classification correspond to errors of $\sim100$ K, which can change masses by factors of 50\% or more.  The luminosities depend on the assumed distance, extinction, spectral-type dependent bolometric corrections, and the observed $J-$ band photometry.  The largest uncertainty from those four are the distance and extinction.  We dereddened to the $J-H$ values assumed for the first pass spectral type.  There is a large intrinsic spread in $J-H$.  So based on this large uncertainty, the luminosities could change by 50\% or more, changing the ages by factors of 3 or more.  


\subsection{What can we say, if anything, about the IMF?}

Our survey was \emph{exceptionally} sensitive to very low luminosity substellar objects and yet we did not find them.  To demonstrate the survey depth, Figure \ref{fig_luminosity_dist} shows the distributions of derived luminosities for all 71 of the sources demonstrating M or L spectral types, overplotted with the distributions of the confirmed young objects.  At face value this result would point to an intrinsic paucity of these sources, as expected from the decreasing number densisty with lower mass of the sub stellar initial mass function.  However, as the IMF is decreasing, the number of contaminating sources is increasing as photometric depth increases.  So the \emph{a-priori} probability that any given source is a young brown dwarf goes down as a function of flux.  We only searched 1\% of the sources in our photometric catalog.  It is conceivable that we have missed very low luminosity brown dwarfs.  If we assume the peak of the IMF is around M5 for this off-core region of \emph{Ophiuchus}, and assume a star to brown dwarf ratio of 4, and assume we have detected all 7 of the M5's towards this region, we would expect zero L0 sources \citep{2012ARA&A..50...65L}.  The non-discovery of young L-type sources is consistent with---though not necessarily evidence for---a steeply declining IMF.  We do not find \emph{any} sources with spectral types later than M8.5, despite our ability to detect them.

A true understanding of the stellar and substellar initial mass function (IMF) requires careful consideration of biases and selection effects, so we make no attempt to derive a shape of the IMF from our sample.  For example, survey selection methods that bias against detection of luminous or underluminous sources will directly flow down to an inaccurate derivation of the IMF.  Since our selection methods involved many different factors, it is a bit tricky to disentangle how these effects might have biased the selections towards and away from the IMF.

\begin{figure}[ht!]
\centering
\includegraphics[scale=0.6]{chIMACS/figures/luminosity_histogram}
\caption[Luminosity distibution of this sample and background contaminants]{Distributions of luminosities of all 71 sources demonstrating M or L spectral types, overplotted with the 16 confirmed young stellar objects. \label{fig_luminosity_dist} }
\end{figure}


\subsection{Is there a higher incidence of edge-on disks in BDs?}
One main result of this work is the discovery and identification of two edge-on disks around low-mass sources, out of only 5 disk-bearing sources towards this region of \emph{Ophiuchus}.  Forty percent of the disks are edge-on.  This fraction is higher than the incidence of edge-on disks around higher mass stars.  We interpret this high fraction as tentative evidence that brown dwarf disks have a larger scale height for a given radial position.  This interpretation is consistent with our intuition that the vertical component of gravity should be much lower for a given radial position, and that the disk inclination should be random on the sky.  Of course, the achieved scale height also depends on the degree of turbulence in the disk, so a detailed analysis would have to consider the average temperature and diffusion timescales at a given location in the disk \citep{2012A&A...539A...9M,2009MNRAS.394L.141E}.  However, \emph{on average}, the systematic bias should tend towards higher gravitational scale height for brown dwarf disks, at a given radial distance.

\subsection{Are any of these sources from Upper Sco?}

One of the main potential uncertainties throughout all of this work is the question of whether the young sources belong to the younger \emph{Ophiuchus} or the older \emph{Upper Scorpius}.  Our region sits between these two on the sky, so it's conceivable that some of our detections include a mix of these two populations.  By plotting the projected positions of known objects, we can get an estimate of the surface number density of low mass objects.  Extrapolating roughly from \citet{2006AJ....131.3016S,2012ApJ...758...31L} we estimate that about 2$-$4 Upper Sco objects could be in our region.  This number is consistent with the appearance of our HR diagram, in which 4 objects sit substantially lower than the 10 Myr isochrone, but above the 100 Myr isochrone.  The slightly older and more distant population of Upper Sco would be expected to occupy a region in this vicinity.  The fact that the disks are more evolved is consistent with an interpretation of an older, $\sim5-11$ Myr population.  For example, the low-mass binary brown dwarf OPH1622-2405 discovered in this region as part of the \citet{allers06} survey was shown by \citet{2007ApJ...659.1629L} to have Na~I line strengths more similar to members of Upper Scorpius than Taurus ($\sim$1 Myr).  

%------------------------------------------------------------------------------------------
\section{Conclusions}
%------------------------------------------------------------------------------------------

We have reported on a broad-band survey aimed at the discovery of young brown dwarfs and very low mass stars towards the \emph{Ophiuchus} star forming region.  We present the selection, spectral analysis, discovery, spectral types, luminosities, and physical characterization of a population of 12 young members, and 3 candidates of an off-core region towards \emph{Ophiuchus}.  Five of the 12 members show disks, while the other 7 members and 3 candidates show photospheric mid-IR colors.  All members show strong evidence for youth through their gravity-sensitive spectral features.  Specifically the Na~I lines are comparable to 1-3 Myr Taurus and Cha I members reported by Luhman and collaborators.  We interpret the majority of these photospheric young sources towards \emph{Ophiuchus} as Class III analogs- diskless young brown dwarfs that have already lost--or never had--disks.  The prevalence of these sources is tantalizing evidence for a shorter-lived disk lifetime for very low mass stars and brown dwarfs than had been previously thought.  This shorter disk lifetime for brown dwarfs would put brown dwarf disk lifetimes more in line with their higher mass T-Tauri star counterparts.  We also take the prevalence of Class III analogs in this study compared to other studies as evidence for a possible bias in selection for sources with evidence for mid-IR excess.  This bias would manifest as an underestimated brown dwarf to star number density ratio, and an over-estimated disk lifetime.  Our key tool in avoiding this bias is the introduction of a custom filter that isolates late type sources from the large number of reddened contaminants.  We caution that contamination from Upper Scorpius could decrease the effect size seen here, but anticipate changes of merely a few objects, based on the low surface number density of Upper Sco.

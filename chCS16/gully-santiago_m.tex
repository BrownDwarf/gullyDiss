\chapter{Confirmation and characterization of young disk-bearing brown dwarfs and sub-brown dwarfs}
\label{ch_CS16}

\section{Introduction}
Star and planet and formation theories predict the formation of substellar mass objects down to near the opacity limit fragmentation mass, with an abundance of brown dwarfs and sub brown dwarfs ejected from multiple systems \citep{bate02,bate09}.  Observations of these failed stars are important probes for the study of star and planet formation, and may be the best method for directly observing planetary mass objects with current technology.  Current moderate resolution ($R\sim2000$) spectrographic instrumentation on large telescopes is capable of detecting both young massive planets ejected from their stellar hosts and the lowest mass young free-floating products of the star formation process \citep{2005ApJ...620L..51L}.  These objects still represent a challenge to identify and characterize given the low source luminosities, relatively large distances ($\gtrsim 100$ pc) to the nearest young star forming regions, presence of variable extinction ($0 < A_{V} < 35$), and the high level of confusion with background and foreground sources.  \citet{allers06} published a list of 19 candidate young very low mass stars, brown dwarfs, and sub brown dwarfs (\emph{i.e.} free floating planetary mass objects) in portions of the Ophiuchus, Lupus I, and Chamaeleon II star forming regions with extinction $A_{V} < 7.5$.  \citeauthor{allers06} selected sources based on near- and mid- IR colors indicative of disk bearing young objects with M and L spectral types.  \citet{allers07} present $R=300$ near-IR spectroscopy of six objects from the \citet{allers06} list, confirming the youth and cool temperatures of all six.  Here we present near-IR ($\lambda=1.0-2.5 \,\mu m$) spectroscopy on 17/19 sources observed with SpeX on NASA's IRTF \citep{rayner03, cushing04, vacca03} in SXD mode (0.8$-$2.4 $\mu$m) with $R\sim1200$ and GNIRS \citep{elias1998,elias06} in cross dispersed mode on Gemini with $R\sim1700$.  The spectra confirm the low surface gravity (and therefore youth) and the M$-$L spectral types of these young objects for all but one source which appears to be extra-galactic \citep{rayjay06}.  The now confirmed sample is an important tool to extend the study of star formation and disk evolution down to near planetary masses.  Additionally, this exceptionally low contamination fraction lends credibility to the selection criteria.  Here we restrict our focus to the characterization of the objects through their near-IR spectra.  In future work, we will present a detailed analysis of the objects and their disks, and the implications for disks around the lowest mass sub-stellar objects.

\section{Spectral Typing}
We determine the spectral types for our sources using the H-band index that quantifies the spectral slope of the H$_{2}$O absorption feature in the broad wavelength range 1.492-1.560$\, \mu$m and is calibrated using optically spectral typed objects of M5-L5 \citep{allers07}.  As a consistency check, we also employ the \citet{slesnick04} $J$-band index which captures the increasing strength of the temperature-sensitive water absorption feature at $1.34 \,\mu$m; the index is defined for types M0-T0.  Both indices are insensitive to surface gravity \citep{slesnick04,allers07} insofar as the indices reproduce the spectral types for standard dwarfs and giants over the spectral type ranges for which they are defined.

The \citet{allers07} $H-$band index requires dereddened spectra, since increased extinction will steepen the slope that the index attempts to quantify.  We employ an iterative technique to derive the extinction and spectral type, specifically using the $J-$ and $H-$ band photometry and spectroscopy.  Our first step is to estimate the extinction from the observed ($J-H$) color, the intrinsic ($J-H$)$_{0}$ \citep{patten06}, and the reddening law of \citet{fitzpatrick1999}.  We assume ($J-H$)$_{0}$=0.6, which is accurate for types M5-M9, rising to $\sim1.08$ at L2 \citep{patten06}.  Next, we derive the spectral type from the \citet{allers07} index, having coarsely dereddened the spectra.  We then refine the estimate for ($J-H$)$_{0}$ with our improved estimate for the spectral type.  We typically iterate this procedure two times, by which the process converges.  Table 1 lists the calculated spectral types from both the $J-$ and $H-$ band indices, and the derived spectral type.  It is instructive to note that the $J-$ and $H-$band indices have a reciprocal dependence on reddening: an underestimation of the extinction would make one conclude a later spectral type with the $H-$ index, but an earlier type for the $J-$band index.  Because of the reciprocal dependence on reddening of our two spectral typing indices, we are confident that our derived spectral types are robust against reddening.  For sources with types earlier than M5, the $H-$ band index is undefined, and so we adopt only the $J-$ band index.  For other sources we derive the spectral type as the average of the $J-$band index and $H-$band index (when both are available), weighted to the value with higher confidence.  The errors in the spectral types are reported from visual comparison of our dereddened spectra to standards with known spectral types: a typical spectrum is firmly sandwiched between two standards of $\pm1$ spectral sub class, based on the overall shape in $J-$, $H-$, and $K-$ bands.

\section{Luminosities}
To compute the luminosities of our sources, we apply a bolometric correction (BC$_J$) to their dereddened $J$-band magnitudes.  We converted BC$_K$ to BC$_J$ for M1--L3 field dwarfs in \citep{Golimowski04} using $K$-band (MKO system) photometry reported therein and $J$-band photometry from the 2MASS PSC.  The spectral type vs. BC$_J$ relation is \footnote{SpT is the numerical translation of SpT: M1--M9=1--9 and L0--L3=10--13} : $BC_J=1.60 + 0.10 \times SpT - 0.007 \times SpT^2$ with small ($\sim$0.08 mag) dispersion about the second order fit.

\smallskip
\begin{center}
{\small
\begin{table}
\begin{tabular}{ccccccc}
\hline
\noalign{\smallskip}
Source & $A_V$ & SpT from A07 & SpT s& SpT a & Derived SpT & $\log L/L_{\odot}$\\
\# &  mag &  & &  &  &\\
\noalign{\smallskip}
\hline
\noalign{\smallskip}
1	&	2.2	$\pm$	0.5	& 9 &	11.4	&	12	&	11.5	$\pm$1	&	-3.3	\\
2	&	10.2	$\pm$	0.4	& 6 &	5.9	&	6	&	6	$\pm$1	&	-1.3	\\
4	&	4.2	$\pm$	0.3	& &	2.1	&	-	&	2	$\pm$1	&	-0.8	\\
5	&	2.8	$\pm$	0.5	& 11&	13.3	&	8.3	&	11	$\pm$2	&	-3.2	\\
6	&	4.3	$\pm$	0.3	& &	4.8	&	-	&	5	$\pm$1	&	-1.2	\\
7	&	2.3	$\pm$	0.4	& &	6.7	&	5.7	&	6	$\pm$1	&	-1.9	\\
8	&	2.8	$\pm$	0.4	& &	2.8	&	-	&	3	$\pm$1	&	-1.2	\\
9	&	6.6	$\pm$	0.4	& &	5.2	&	5.9	&	5.5	$\pm$1	&	-1.4	\\
10	&	4.7	$\pm$	0.3	& &	3.1	&	-	&	3	$\pm$1	&	-0.6	\\
11n	&	0.4	$\pm$	0.6	& 7 &	5.2	&	7	&	7	$\pm$1	&	-2.5	\\
11s	&	0.2	$\pm$	0.7	& 8 &	9.7	&	9.2	&	9	$\pm$1	&	-2.8	\\
13	&	3.3	$\pm$	0.3	& &	6.1	&	6.2	&	6	$\pm$1	&	-1.8  \\
14	&	6.1	$\pm$	0.4	& 7 &	-	&	8.1	&	8	$\pm$1	&	-2.3	\\
15	&	10.2	$\pm$	0.4	& &	2.3	&	-	&	2	$\pm$1	&	-1.1	\\
16n	&	3.7	$\pm$	0.5	& &	3	&	5.2	&	3	$\pm$1	&	-0.8	\\
16s	&	3.7	$\pm$	0.5	& &	3.9	&	5.1	&	4	$\pm$1	&	-0.9	\\
17	&	0	$\pm$	0.4	& &	12.1	&	13.2	&	12.5	$\pm$2	&	-3.3	\\
18	&	0.2	$\pm$	0.5	& &	6.1	&	7.5	&	7	$\pm$1	&	-2.4	\\
19	&	0	$\pm$	0.3	& &	4.3	&	5.5	&	4.5	$\pm$1	&	-1.7	\\
\noalign{\smallskip}
\hline
\end{tabular}
 \caption[Derived extinction, spectral type, and luminosity for Allers \emph{et al.} 2006 sources]{Derived extinction, spectral type, and luminosity.  Source \# is the number from Tables $3-6$ in \citet{allers06}, SpT from A07 is the spectral type reported in \citet{allers07}, SpT s and SpT a are the spectral types derived here using the prescription of \citet{slesnick04} and \citet{allers07}, respectively.  The derived spectral type is an average of the two types when possible, weighted to the value of higher confidence.  For source \#16 n and s, we adopt only the \citet{slesnick04} index.  The uncertainties of the extinction are propagated from the uncertainty in $(J-H)_{0}$, and an uncertainty of 1 spectral subclass in the spectral type.}
 \end{table}
}
\end{center}

\begin{figure}[!ht]
\plottwo{chCS16/gully-santiago_m_fig1}{chCS16/gully-santiago_m_fig2}
\caption[Spectral classification with spectral type standards and spectra of dwarfs and giants]{\emph{Left-} Spectral typing comparison for source 13 (derived SpT= M6), with spectral type comparison objects from the SpeX archive \citep{rayner09,cushing05}: Gl51 (M5V), and Gl644C (M7V).  \emph{Right-} Detail of the $\mathrm{Na\, I}$ line for source 13, with surface gravity comparison objects with identical spectral types, also from the SpeX archive: Gl406 (M6V) and HD196610 (M6 III).  Source 13 is intermediate in surface gravity as demonstrated by the gravity sensitive index $\mathrm{Na\, I}$.}
\end{figure}

\begin{figure}[!ht]
\plotone{chCS16/gully-santiago_m_fig3}
\caption[Gravity dependence of the $\mathrm{Na\, I}$ line]{Gravity dependence of the $\mathrm{Na\, I}$ line.  Dwarfs and giants from the SpeX spectral atlas are shown as red diamonds and green triangles respectively.  The blue squares show the sources from this work.  Compare to figure 35 of \citet{rayner09}.}
\end{figure}

\section{Confirmation of Youth}
Once the sources were established as cool stars or sub-stellar objects, their youth was probable based on the presence of mid-IR excess \citep{allers06} indicative of circumstellar disks.  We improved the case for youth by quantifying gravity sensitive spectral lines.  The $\mathrm{Na\, I}$ $\lambda=1.14 \,\mu$m equivalent width is sensitive to surface gravity \citep{rayner09}, showing a clear dichotomy in the spectra of dwarfs and giants after about M4, specifically with the dwarfs demonstrating increasing equivalent widths up to 15 $\AA$, while the giants exhibit equivalent widths of about 1 $\AA$.  The right panel of figure 1 shows a detail of the  $\mathrm{Na\, I}$ line and Figure 2 shows the NaI equivalent width as a function of spectral type, with the independently measured equivalent widths of dwarfs and giants from the SpeX spectral atlas \citep{rayner09, cushing05}, and the intermediate widths of young substellar objects in this sample.  The increased resolution over the Allers et al. \citep{allers07} study was critical for measuring the equivalent widths of lines since, for example, \citet{allers07} employed a flux ratio index to quantify the degree of $\mathrm{Na\, I}$ absorption because the line blending is too severe at $R\sim$300.  With $R\sim1200$, one can isolate the $\mathrm{Na\, I}$ line well enough to notice that its strength is intermediate to dwarfs and giants, and not merely less than dwarfs.  For future studies targeting diskless young sources, spectra at resolution similar to this study will be best for identifying weak gravity sensitive lines like $\mathrm{Mg\, I}$, $\mathrm{Al\, I}$, and $\mathrm{Na\, I}$ \citep{rayner09} for the purpose of assigning youth to the sources.  
